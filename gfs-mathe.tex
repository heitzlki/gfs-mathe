\documentclass[12pt,a4paper]{article}
\usepackage[utf8]{inputenc}
\usepackage[T1]{fontenc}
\usepackage[german]{babel} %deutsche Sprache
\usepackage{amsmath} %für Formeln
\usepackage{amsfonts} %Für Text in Formeln
\usepackage{amssymb} %Für mathematische Symbole
\usepackage{graphicx} %Für Bilder

\usepackage[absolute]{textpos}%Grafiken Positionieren

\usepackage{import} %für .pdf_tex Grafiken
\usepackage{color} %für .pdf_tex Grafiken
\usepackage{transparent} %für .pdf_tex Grafiken
\graphicspath{{assets/main/}} %für .pdf_tex Grafiken
\usepackage{wrapfig} %Um Text neben Bildern zu positionieren

\usepackage{multicol} %Für mehr-spaltigen Text

\usepackage{hyperref} %Für Links wie z.B Abbildung X
\hypersetup{
	hidelinks
} %Um Link-Stile zu deaktivieren

\usepackage{subcaption}
\captionsetup{compatibility=false}

\usepackage[left=2cm,right=2cm,top=2cm,bottom=2cm]{geometry}  %Dokument Rand etc. einstellen
%\renewcommand{\familydefault}{\sfdefault} %sans-serif Schriftart
\title{Mit dem Sinus modellieren}
\author{Kirill Heitzler}
\date{\today}

\begin{document}

\maketitle
\begin{textblock}{210}(-1,4)
	\begin{figure}[h!] 
		\input{assets/main/Title.path.pdf_tex} %Grafik muss 220mm breit sein
	\end{figure}
\end{textblock}
\newpage

\tableofcontents

\newpage

\section{Rückblick}

\subsection{Rechtwinkliges Dreieck - Beschriftung} 
\begin{figure}[hb!]
    \centering
    \def\svgwidth{200px}
    \input{assets/main/rechtwinkliges_dreieck.path.pdf_tex}
    \caption{Rechtwinkliges Dreieck}
  	\label{fig:rechtwinkliges_dreieck}
\end{figure}
\noindent
Das Rechtwinklige Dreieck wird folgendermaßen wie in \autoref{fig:rechtwinkliges_dreieck} beschriftet. \\
Die Ecken werden mit den Buchstaben A, B, C gegen den Uhrzeigersinn bei A angefangen beschriftet. \\
Die Winkel $\alpha$, $\beta$, $\gamma$ werden in die Ecken der entsprechenden Buchstaben A, B, C gesetzt. \\
Die anliegende Kathete zu Winkel $\alpha$ wir 'Ankathete von $\alpha$' genannt und die Kathete gegenüber von Alpha wird 'Gegenkathete von $\alpha$'´ genannt. \\
Die Hypothenuse liegt gegenüber des Rechten Winkel $\gamma$.

\subsection{Der Sinus}
\textbf{Definition:} In einem rechtwinkligen Dreieck nennt man zu einem Winkel $\alpha$ des Dreiecks das Streckenverhältnis  
\begin{align}
 \sin(\alpha) &= \dfrac{\text{Gegenkathete von $\alpha$}}{\text{Hypothenuse}}
\end{align}
den \textbf{Sinus von $\alpha$}
\begin{figure}[hb!]
    \centering
    \def\svgwidth{200px}
    \input{assets/main/rechtwinkliges_dreieck_sinus.path.pdf_tex}
    \caption{Rechtwinkliges Dreieck}
  	\label{fig:rechtwinkliges_dreieck_sinus}
\end{figure}
\newpage

\subsection{Der Sinus - Beispiel Aufgabe}
\textbf{Gegenkathete von $\alpha$ mithilfe des Sinus berechnen}: \\
\textbf{Aufgabe:} Berechne die Höhe des Freiburger Münsters. Das rechtwinklige Dreieck in \autoref{fig:rechtwinkliges_dreieck_am_muenster} besitzt einen rechten Winkel(90°), die Hyptenuse 164,05 Meter und die Winkelweite des Winkels $\alpha$ mit 45°. Berechne die Gegenkathete von $\alpha$ namen's x.
\begin{figure}[hb!]
    \centering
    \def\svgwidth{300px}
    \input{assets/main/rechtwinkliges_dreieck_am_muenster.path.pdf_tex}
    \caption{Rechtwinkliges Dreieck am Münster}
  	\label{fig:rechtwinkliges_dreieck_am_muenster}
\end{figure}
\textbf{Rechnung:}
\begin{align}
 \sin(\alpha) &= \dfrac{\text{Gegenkathete von $\alpha$}}{\text{Hypothenuse}} \tag{1}\\
 \sin(45°) &= \dfrac{x}{164,05m} & |\cdot 164,05m \\
 \sin(45°) \cdot 164,05m &= x\\
 x &\cong 116m
\end{align}
\textbf{Antwort:} Die Gegenkathete von $\alpha$ beträgt etwa 116 Meter, somit ist das Münster auch etwa 116 Meter groß.
\newpage
\subsection{Der Kosinus und der Tangens}
\textbf{Sinus von $\alpha$:}
\begin{align}
 \sin(\alpha) &= \dfrac{\text{Gegenkathete von $\alpha$}}{\text{Hypothenuse}} \tag{1}
\end{align}
\begin{figure}[hb!]
    \centering
    \def\svgwidth{200px}
    \input{assets/main/rechtwinkliges_dreieck_sinus.path.pdf_tex}
    \caption{Rechtwinkliges Dreieck}
  	\label{fig:rechtwinkliges_dreieck_sinus}
\end{figure}
\\
\textbf{Cosinus von $\alpha$:}
\begin{align}
 \cos(\alpha) &= \dfrac{\text{Ankathete von $\alpha$}}{\text{Hypothenuse}} \tag{1}
\end{align}
\begin{figure}[hb!]
    \centering
    \def\svgwidth{200px}
    \input{assets/main/rechtwinkliges_dreieck_cosinus.path.pdf_tex}
    \caption{Rechtwinkliges Dreieck}
  	\label{fig:rechtwinkliges_dreieck_cosinus}
\end{figure}
\\
\textbf{Tangens von $\alpha$:}
\begin{align}
 \tan(\alpha) &= \dfrac{\text{Gegenkathete von $\alpha$}}{\text{Ankathete von $\alpha$}} \tag{1}
\end{align}
\begin{figure}[hb!]
    \centering
    \def\svgwidth{200px}
    \input{assets/main/rechtwinkliges_dreieck_tangens.path.pdf_tex}
    \caption{Rechtwinkliges Dreieck}
  	\label{fig:rechtwinkliges_dreieck_tangens}
\end{figure}
\newpage
\section{Einheitskreis}
\subsection{Einheitskreis - Beispiel}

\textbf{Aufgaben-Text:} Auf einem kresiförmigen Koordinatensystem eines Radarschirms \autoref{fig:radar} wird die Lage von zwei Schiffen durch die Entfernung zum Hafen(0) und durch den Kurs gegenüber der x-Achse beschrieben. \\
\textbf{Aufgabe:} Ein Schiff \textbf{A} ist mit dem Kurs \textbf{30°} gegenüber der x-Achse \textbf{einen Kilometer} weit gefahren. Welche Koordinaten im \textbf{x-y-Kooradinatensystem} hat es? Welche Koordinaten hat das Schiff \textbf{B}, das mit dem Kurs \textbf{75°} textbf{einen Kilometer} weit gefahren ist?
\begin{figure}[hb!]
    \centering
    \def\svgwidth{250px}
    \input{assets/main/Radar.path.pdf_tex}
    \caption{Radar}
  	\label{fig:radar}
\end{figure}
\\
\noindent
\textbf{Lösung:} 
\begin{wrapfigure}[9]{r}{0.5\textwidth}
	    \def\svgwidth{250px}
    \input{assets/main/radar_loesung.path.pdf_tex}
    \caption{Radar Lösung}
  	\label{fig:radar_loesung}
\end{wrapfigure}
\\
Das Schiff \textbf{A} mit dem Kurs \textbf{30°} befindet sich auf der x-Achse: etwa \textbf{0,86 Kilometer} und y-Achse: \textbf{0,5 Kilometer}. Also auf dem Punkt \textbf{A(0,86|0,5)} \\  \\
Das Schiff \textbf{B} mit dem Kurs \textbf{75°} befindet sich auf der x-Achse: etwa \textbf{0,25 Kilometer} und y-Achse: \textbf{0,96 Kilometer}. Also auf dem Punkt \textbf{A(0,25|0,96)}

\newpage
\subsection{Der Sinus und Kosinus am Einheitskreis}
\noindent
Dreiecke mit der \textbf{Hypotenusenlänge 1} kann man in einem Koordinatensystem auf folgenden Weise darstellen:
\begin{multicols}{2}
\paragraph{1. }
Die Endpunkte der \textbf{Hypotenuse} sind der Ursprung O und ein Punkt \textbf{P}, der auf einem Kreis O mit dem \textbf{Radius 1} liegt. Diesen Kreis nennt man den \textbf{Einheitskreis}.
\paragraph{2. }
Die Ecke mit dem rechten Winkel liegt auf der \textbf{x-Achse senkrecht unter P}. Der Punkt P hat somit Koordinaten \textbf{P(cos($\alpha$)|sin($\alpha$))}
\end{multicols}
\begin{figure}[hb!]
	\centering
    \def\svgwidth{250px}
    \input{assets/main/sin_cos_einheitskreis.path.pdf_tex}
    \caption{Sinus und Kosinus am Einheitskreis}
  	\label{fig:sin_cos_einheitskreis}
\end{figure}
\newpage

\subsection{Beziehungen zwischen Sinus, Kosinus und Tangens}
\paragraph{1.} 
\newcommand{\smallImageSize}{130px} %Size for the Images saved in a variable
\begin{figure}[h!]
  \centering
  \begin{subfigure}[b]{0.3\linewidth}
	\def\svgwidth{\smallImageSize}
    \input{assets/main/0_alpha_90.path.pdf_tex}
    \caption{$0° < \alpha < 90°$}
    \label{fig:0_alpha_90}
  \end{subfigure}
  \begin{subfigure}[b]{0.3\linewidth}
  \def\svgwidth{\smallImageSize}
    \input{assets/main/sin_null_cos_null_einheitskreis.path.pdf_tex}
	\caption{$sin(0°) = 0, cos(0°) = 1$}
	\label{fig:sin_null_cos_null_einheitskreis}
  \end{subfigure}
  \begin{subfigure}[b]{0.3\linewidth}
  \def\svgwidth{\smallImageSize}
    \input{assets/main/sin_neunzig_cos_neunzig_einheitskreis.path.pdf_tex}
    \caption{$sin(90°) = 1, cos(90°) = 1$}
    \label{fig:sin_neunzig_cos_neunzig_einheitskreis}
  \end{subfigure}
  \caption{Beziehung 1}
  \label{fig:beziehung_eins}
\end{figure}

\noindent Für $0° < \alpha < 90°$ nimmt $sin(\alpha)$ mit wachsendem $\alpha$ zu und $cos(\alpha)$ ab(\autoref{fig:0_alpha_90}). \\
$sin(0°) = 0, cos(0°) = 1$ (\autoref{fig:sin_null_cos_null_einheitskreis}), $sin(90°) = 1, cos(90°) = 0$ (\autoref{fig:sin_neunzig_cos_neunzig_einheitskreis}).

\paragraph{2.} %Layout anpassen
\begin{wrapfigure}[4]{r}{0.3\textwidth}
	\def\svgwidth{\smallImageSize}
    \input{assets/main/radar_loesung.path.pdf_tex}
    \caption{Einheitskreis Dreieck Satz des Pythagoras}
  	\label{fig:einheitskreis_dreieck_pythagoras}
\end{wrapfigure}
Wendet man auf das im Einheitskreis dargestellte Dreieck den Satz des Pythagoras an(\autoref{fig:einheitskreis_dreieck_pythagoras}), so erhält man den für jede Winkelweite gültigen Zusammenhang \\
$sin^2(\alpha) + cos^2(\alpha) = 1$.
\paragraph{3.}
\begin{wrapfigure}[5]{l}{0.3\textwidth}
	\def\svgwidth{\smallImageSize}
    \input{assets/main/radar_loesung.path.pdf_tex}
    \caption{sin(90° - $\alpha$), cos(90° - $\alpha$)}
  	\label{fig:sin_neunzig_minus_alpha_cos_neunzig_minus_alpha}
\end{wrapfigure}
In \autoref{fig:sin_neunzig_minus_alpha_cos_neunzig_minus_alpha} sieht man: \\
$sin(90° - \alpha) = x = cos(\alpha)$ und \\
$cos(90° - \alpha) = y = sin(\alpha)$
\paragraph{4.}
Ebenfalls in \autoref{fig:sin_neunzig_minus_alpha_cos_neunzig_minus_alpha}:\\
$tan(\alpha) = \frac{y}{x} = \frac{sin(\alpha)}{\cos(\alpha)}$.
\newpage
\subsection{Einheitskreis - Definition}
\textbf{Definition:} Es gilt:
\paragraph{}
$sin^2(\alpha) + cos^2(\alpha) = 1$
\paragraph{}
$sin(90° - \alpha) = sin(\alpha)$
\paragraph{}
$tan(\alpha) = \frac{sin(\alpha)}{\cos(\alpha)}$,$\alpha \neq 90°$, weil: 
$tan(90)=\frac{sin(90)}{cos(90)} = \frac{1}{0} = !$.
\subsection{Einheitskreis - Aufgabe}

\newpage
\section{Mit dem Sinus modellieren}
\subsection{Mit dem Sinus modellieren - Beispiel}
\subsection{Mit dem Sinus modellieren - Wertetabelle}
\subsection{Mit dem Sinus modellieren - Zeichnung}

\subsection{Mit dem Sinus modellieren - Winkel $\alpha$ mit 0°...<}
\subsection{Mit dem Sinus modellieren - Zeichnung}
\subsection{Mit dem Sinus modellieren - Zeichnung}

\subsection{Sinusfunktion im Gradmaß - Definition}
\subsection{Mit dem Sinus modellieren - Aufgabe}

\section{Zusammenfassung}

\section{Anwendungsbeispiele / Weiter Anwendungen}

\section{Quellen}
Freiburger Münster - \url{https://freiburg-schwarzwald.de/fotos06feb/freiburg12-060227.jpg}
\end{document} \\
Lambacher Sweizer 9 - Mathematik Buch
