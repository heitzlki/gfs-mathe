\documentclass{standalone}
\begin{document}

\subsection{Funktion f mit f($\alpha$)}

Ordnet man jedem Winkel $\alpha$ mit 0$^\circ$  $\leq \alpha \leq$ 360$^\circ$  seinen Sinuswert zu, so erh{"a}lt man eine Funktion f mit f($\alpha$) = sin($\alpha$).\\
Man kann mithilfe des Graphen von f(\autoref{fig:sinuswelle}) zu gegebenem Winkel den Sinuswert n{"a}herungsweise ablesen oder n{"a}herungsweise Winkel mit vorgegebenem Sinuswert ermitteln.

\begin{figure}[hb!]
  \center
  \def\svgwidth{500px}
  \input{tmp/sinuswelle.path.svg.pdf_tex}
  \caption{$f(\alpha) = sin(\alpha)$}
  \label{fig:sinuswelle}
\end{figure}

\end{document}