\documentclass{standalone}
\begin{document}

\subsection{Mit dem Sinus Modellieren}
Ordnet man jedem Winkel $\alpha$ mit $0^\circ \leq \alpha \leq 360^\circ$  seinen Sinuswert zu, so erh{"a}lt man die Sinusfunktion im Gradma{\ss} $f$ mit $f(\alpha) = sin(\alpha)$. Tr{"a}gt man die Werte der Sinusfunktion im Gradma{\ss} in ein entsprechendes Koordinatensystem erh{"a}lt man den Grafphen von $f$ (\autoref{fig:sinuswelle_two}).

\begin{figure}[hb!]
  \center
  \def\svgwidth{450px}
  \input{tmp/sinuswelle.path.svg.pdf_tex}
  \caption{$f(\alpha) = \sin(\alpha)$}
  \label{fig:sinuswelle_two}
\end{figure}

\end{document}