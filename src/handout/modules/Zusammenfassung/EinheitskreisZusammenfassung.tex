\documentclass{standalone}
\begin{document}

\subsection{Einheitskreis}
\noindent Die Endpunkte eines Dreickecks mit der Hypotenusenl{"a}nge 1 bilden den Ursprung 0 und einen Punkt P, der auf einem Kreis um 0 mit dem Radius 1 liegt und den Einheitskreis bildet.
\vspace{-0.3cm}  %%%%% Vertical spacing
\begin{figure}[hb!]
  \center
  \def\svgwidth{150px}
  \input{tmp/0_alpha_90_360.path.svg.pdf_tex}
  \caption{Einheitskreis}
  \label{fig:0_alpha_90_360}
\end{figure}
\\\noindent Die Gegenkathete l{"a}sst sich mit $\sin(\alpha)$ und die Ankathete mit $\cos(\alpha)$ berechnen. 
\vspace{-0.3cm}  %%%%% Vertical spacing
\begin{figure}[hb!]
	\centering
	\def\svgwidth{150px}
	\input{tmp/sin_cos_einheitskreis.path.svg.pdf_tex}
	\caption{Sinus und Kosinus am Einheitskreis}
	\label{fig:sin_cos_einheitskreis}
\end{figure}
\end{document}