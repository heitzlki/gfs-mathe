\section{Rückblick}

\subsection{Rechtwinkliges Dreieck - Beschriftung}
\begin{figure}[hb!]
  \centering
  \def\svgwidth{200px}
  \input{assets/main/rechtwinkliges_dreieck.path.pdf_tex}
  \caption{Rechtwinkliges Dreieck}
  \label{fig:rechtwinkliges_dreieck}
\end{figure}
\noindent
Das Rechtwinklige Dreieck wird folgendermaßen wie in \autoref{fig:rechtwinkliges_dreieck} beschriftet. \\
Die Ecken werden mit den Buchstaben A, B, C gegen den Uhrzeigersinn bei A angefangen beschriftet. \\
Die Winkel $\alpha$, $\beta$, $\gamma$ werden in die Ecken der entsprechenden Buchstaben A, B, C gesetzt. \\
Die anliegende Kathete zu Winkel $\alpha$ wir 'Ankathete von $\alpha$' genannt und die Kathete gegenüber von Alpha wird 'Gegenkathete von $\alpha$'´ genannt. \\
Die Hypothenuse liegt gegenüber des Rechten Winkel $\gamma$.

\subsection{Der Sinus}
\textbf{Definition:} In einem rechtwinkligen Dreieck nennt man zu einem Winkel $\alpha$ des Dreiecks das Streckenverhältnis
\begin{align}
  \sin(\alpha) & = \dfrac{\text{Gegenkathete von $\alpha$}}{\text{Hypothenuse}}
\end{align}
den \textbf{Sinus von $\alpha$}
\begin{figure}[hb!]
  \centering
  \def\svgwidth{200px}
  \input{assets/main/rechtwinkliges_dreieck_sinus.path.pdf_tex}
  \caption{Rechtwinkliges Dreieck}
  \label{fig:rechtwinkliges_dreieck_sinus}
\end{figure}
\newpage

\subsection{Der Sinus - Beispiel Aufgabe}
\textbf{Gegenkathete von $\alpha$ mithilfe des Sinus berechnen}: \\
\textbf{Aufgabe:} Berechne die Höhe des Freiburger Münsters. Das rechtwinklige Dreieck in \autoref{fig:rechtwinkliges_dreieck_am_muenster} besitzt einen rechten Winkel(90°), die Hyptenuse 164,05 Meter und die Winkelweite des Winkels $\alpha$ mit 45°. Berechne die Gegenkathete von $\alpha$ namen's x.
\begin{figure}[hb!]
  \centering
  \def\svgwidth{300px}
  \input{assets/main/rechtwinkliges_dreieck_am_muenster.path.pdf_tex}
  \caption{Rechtwinkliges Dreieck am Münster}
  \label{fig:rechtwinkliges_dreieck_am_muenster}
\end{figure}
\textbf{Rechnung:}
\begin{align}
  \sin(\alpha)            & = \dfrac{\text{Gegenkathete von $\alpha$}}{\text{Hypothenuse}} \tag{1}                  \\
  \sin(45°)               & = \dfrac{x}{164,05m}                                                   & |\cdot 164,05m \\
  \sin(45°) \cdot 164,05m & = x                                                                                     \\
  x                       & \cong 116m
\end{align}
\textbf{Antwort:} Die Gegenkathete von $\alpha$ beträgt etwa 116 Meter, somit ist das Münster auch etwa 116 Meter groß.
\newpage
\subsection{Der Kosinus und der Tangens}
\textbf{Sinus von $\alpha$:}
\begin{align}
  \sin(\alpha) & = \dfrac{\text{Gegenkathete von $\alpha$}}{\text{Hypothenuse}} \tag{1}
\end{align}
\begin{figure}[hb!]
  \centering
  \def\svgwidth{200px}
  \input{assets/main/rechtwinkliges_dreieck_sinus.path.pdf_tex}
  \caption{Rechtwinkliges Dreieck}
  \label{fig:rechtwinkliges_dreieck_sinus}
\end{figure}
\\
\textbf{Cosinus von $\alpha$:}
\begin{align}
  \cos(\alpha) & = \dfrac{\text{Ankathete von $\alpha$}}{\text{Hypothenuse}} \tag{1}
\end{align}
\begin{figure}[hb!]
  \centering
  \def\svgwidth{200px}
  \input{assets/main/rechtwinkliges_dreieck_cosinus.path.pdf_tex}
  \caption{Rechtwinkliges Dreieck}
  \label{fig:rechtwinkliges_dreieck_cosinus}
\end{figure}
\\
\textbf{Tangens von $\alpha$:}
\begin{align}
  \tan(\alpha) & = \dfrac{\text{Gegenkathete von $\alpha$}}{\text{Ankathete von $\alpha$}} \tag{1}
\end{align}
\begin{figure}[hb!]
  \centering
  \def\svgwidth{200px}
  \input{assets/main/rechtwinkliges_dreieck_tangens.path.pdf_tex}
  \caption{Rechtwinkliges Dreieck}
  \label{fig:rechtwinkliges_dreieck_tangens}
\end{figure}