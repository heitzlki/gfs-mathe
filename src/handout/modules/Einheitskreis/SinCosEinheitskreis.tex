\documentclass{standalone}
\begin{document}

\subsection{Der Sinus und Kosinus am Einheitskreis}
\noindent
Dreiecke mit der \textbf{Hypotenusenl{"a}nge 1} kann man in einem Koordinatensystem auf folgenden Weise darstellen:
\begin{multicols}{2}
	\paragraph{1. }
	Die Endpunkte der \textbf{Hypotenuse} sind der Ursprung O und ein Punkt \textbf{P}, der auf einem Kreis O mit dem \textbf{Radius 1} liegt. Diesen Kreis nennt man den \textbf{Einheitskreis}.
	\paragraph{2. }
	Die Ecke mit dem rechten Winkel liegt auf der \textbf{x-Achse senkrecht unter P}. Der Punkt P hat somit Koordinaten \textbf{P(cos($\alpha$)|sin($\alpha$))}
\end{multicols}
\begin{figure}[hb!]
	\centering
	\def\svgwidth{250px}
	\input{tmp/sin_cos_einheitskreis.path.svg.pdf_tex}
	\caption{Sinus und Kosinus am Einheitskreis}
	\label{fig:sin_cos_einheitskreis}
\end{figure}

\end{document}
