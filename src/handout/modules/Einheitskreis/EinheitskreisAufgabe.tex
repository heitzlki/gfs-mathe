\documentclass{standalone}

\begin{document}

\subsection{Einheitskreis - Aufgabe}
\textbf{Aufgabe:}
$sin(\alpha) = 0,6$. \\

\noindent\textbf{Bestimme:}

\begin{center}
	\normalsize
	\begin{tabular}{ c c c c c }
		\normalsize\textbf{a)} $cos(\alpha)$            &

		\normalsize\textbf{b)} $tan(\alpha)$            &

		\normalsize\textbf{c)} $sin(90^\circ - \alpha)$ &

		\normalsize\textbf{d)} $cos(90^\circ - \alpha)$ &

		\normalsize\textbf{e)} $tan(90^\circ - \alpha)$
	\end{tabular}
\end{center}
\\

\noindent\textbf{a) L{"o}sung:}
\begin{align}
	sin^2(\alpha) + cos^2(\alpha) & = 1 \tag{1}                                    \\
	0,6^2 + cos^2(\alpha)         & = 1                         & |-0,6^2 \tag{2}  \\
	cos^2(\alpha)                 & = 1 - 0,36                  & |\sqrt{} \tag{3} \\
	cos(\alpha)                   & = \sqrt{1 - 0,36}   \tag{4}                    \\
	cos(\alpha)                   & = \sqrt{0,64}   \tag{5}                        \\
	cos(\alpha)                   & = 0,8   \tag{6}
\end{align}

\noindent\textbf{b) L{"o}sung:}
\begin{align}
	tan(\alpha) & = \frac{sin(\alpha)}{cos(\alpha)} \tag{1} \\
	tan(\alpha) & = \frac{0,6}{0,8} = \frac{6}{8} \tag{2}   \\
	tan(\alpha) & = \frac{3}{4} = 0,75 \tag{3}
\end{align}

\noindent\textbf{c) L{"o}sung:}
\begin{align}
	sin(90^\circ  - \alpha) & = cos(\alpha) = 0,8 \tag{1}
\end{align}

\noindent\textbf{d) L{"o}sung:}
\begin{align}
	cos(90^\circ  - \alpha) & = sin(\alpha) = 0,6 \tag{1}
\end{align}

\noindent\textbf{e) L{"o}sung:}
\begin{align}
	tan(90^\circ  - \alpha) & = \frac{sin(90^\circ  - \alpha)}{cos(90^\circ  - \alpha)} = \frac{0,8}{0,6} \tag{1} \\
	tan(90^\circ  - \alpha) & = \frac{8}{6} = \frac{4}{3} \tag{2}
\end{align}

\end{document}
