\documentclass[12pt,a4paper]{article}
\usepackage[utf8]{inputenc}
\usepackage[T1]{fontenc}
\usepackage[german]{babel} %deutsche Sprache
\usepackage{amsmath} %für Formeln
\usepackage{amsfonts} %Für Text in Formeln
\usepackage{amssymb} %Für mathematische Symbole
\usepackage{graphicx} %Für Bilder
\UseRawInputEncoding

\usepackage[absolute]{textpos}%Grafiken Positionieren

\usepackage{import} %für .pdf_tex Grafiken
\usepackage{color} %für .pdf_tex Grafiken
\usepackage{transparent} %für .pdf_tex Grafiken
\graphicspath{{tmp/}} %für .pdf_tex Grafiken
\usepackage{wrapfig} %Um Text neben Bildern zu positionieren

\usepackage{multicol} %Für mehr-spaltigen Text
\usepackage{xcolor,colortbl} %Für z.B Farben in Tabellen

\usepackage{hyperref} %Für Links wie z.B Abbildung X
\hypersetup{
	hidelinks
} %Um Link-Stile zu deaktivieren

\usepackage{adjustbox}%Um Elemente auf Seitenlänge zu bringen
\usepackage{standalone}%Um das Dokument aufzuteilen
\usepackage{ marvosym } %Füt lightning symbol
\usepackage{subcaption}
\captionsetup{compatibility=false}

\usepackage[left=2cm,right=2cm,top=2cm,bottom=2cm]{geometry}  %Dokument Rand etc. einstellen
%\renewcommand{\familydefault}{\sfdefault} %sans-serif Schriftart
\title{Mit dem Sinus modellieren}
\author{Kirill Heitzler}
\date{\today}

\begin{document}

\maketitle
\begin{textblock}{210}(-1,4)
	\begin{figure}[h!]
		\input{tmp/title.path.svg.pdf_tex} %Grafik muss 220mm breit sein
	\end{figure}
\end{textblock}
\newpage

\tableofcontents
\newpage
\section{Rückblick}

\subsection{Rechtwinkliges Dreieck - Beschriftung}
\begin{figure}[hb!]
  \centering
  \def\svgwidth{200px}
  \input{assets/main/rechtwinkliges_dreieck.path.pdf_tex}
  \caption{Rechtwinkliges Dreieck}
  \label{fig:rechtwinkliges_dreieck}
\end{figure}
\noindent
Das Rechtwinklige Dreieck wird folgendermaßen wie in \autoref{fig:rechtwinkliges_dreieck} beschriftet. \\
Die Ecken werden mit den Buchstaben A, B, C gegen den Uhrzeigersinn bei A angefangen beschriftet. \\
Die Winkel $\alpha$, $\beta$, $\gamma$ werden in die Ecken der entsprechenden Buchstaben A, B, C gesetzt. \\
Die anliegende Kathete zu Winkel $\alpha$ wir 'Ankathete von $\alpha$' genannt und die Kathete gegenüber von Alpha wird 'Gegenkathete von $\alpha$'´ genannt. \\
Die Hypothenuse liegt gegenüber des Rechten Winkel $\gamma$.

\subsection{Der Sinus}
\textbf{Definition:} In einem rechtwinkligen Dreieck nennt man zu einem Winkel $\alpha$ des Dreiecks das Streckenverhältnis
\begin{align}
  \sin(\alpha) & = \dfrac{\text{Gegenkathete von $\alpha$}}{\text{Hypothenuse}}
\end{align}
den \textbf{Sinus von $\alpha$}
\begin{figure}[hb!]
  \centering
  \def\svgwidth{200px}
  \input{assets/main/rechtwinkliges_dreieck_sinus.path.pdf_tex}
  \caption{Rechtwinkliges Dreieck}
  \label{fig:rechtwinkliges_dreieck_sinus}
\end{figure}
\newpage

\subsection{Der Sinus - Beispiel Aufgabe}
\textbf{Gegenkathete von $\alpha$ mithilfe des Sinus berechnen}: \\
\textbf{Aufgabe:} Berechne die Höhe des Freiburger Münsters. Das rechtwinklige Dreieck in \autoref{fig:rechtwinkliges_dreieck_am_muenster} besitzt einen rechten Winkel(90°), die Hyptenuse 164,05 Meter und die Winkelweite des Winkels $\alpha$ mit 45°. Berechne die Gegenkathete von $\alpha$ namen's x.
\begin{figure}[hb!]
  \centering
  \def\svgwidth{300px}
  \input{assets/main/rechtwinkliges_dreieck_am_muenster.path.pdf_tex}
  \caption{Rechtwinkliges Dreieck am Münster}
  \label{fig:rechtwinkliges_dreieck_am_muenster}
\end{figure}
\textbf{Rechnung:}
\begin{align}
  \sin(\alpha)            & = \dfrac{\text{Gegenkathete von $\alpha$}}{\text{Hypothenuse}} \tag{1}                  \\
  \sin(45°)               & = \dfrac{x}{164,05m}                                                   & |\cdot 164,05m \\
  \sin(45°) \cdot 164,05m & = x                                                                                     \\
  x                       & \cong 116m
\end{align}
\textbf{Antwort:} Die Gegenkathete von $\alpha$ beträgt etwa 116 Meter, somit ist das Münster auch etwa 116 Meter groß.
\newpage
\subsection{Der Kosinus und der Tangens}
\textbf{Sinus von $\alpha$:}
\begin{align}
  \sin(\alpha) & = \dfrac{\text{Gegenkathete von $\alpha$}}{\text{Hypothenuse}} \tag{1}
\end{align}
\begin{figure}[hb!]
  \centering
  \def\svgwidth{200px}
  \input{assets/main/rechtwinkliges_dreieck_sinus.path.pdf_tex}
  \caption{Rechtwinkliges Dreieck}
  \label{fig:rechtwinkliges_dreieck_sinus}
\end{figure}
\\
\textbf{Cosinus von $\alpha$:}
\begin{align}
  \cos(\alpha) & = \dfrac{\text{Ankathete von $\alpha$}}{\text{Hypothenuse}} \tag{1}
\end{align}
\begin{figure}[hb!]
  \centering
  \def\svgwidth{200px}
  \input{assets/main/rechtwinkliges_dreieck_cosinus.path.pdf_tex}
  \caption{Rechtwinkliges Dreieck}
  \label{fig:rechtwinkliges_dreieck_cosinus}
\end{figure}
\\
\textbf{Tangens von $\alpha$:}
\begin{align}
  \tan(\alpha) & = \dfrac{\text{Gegenkathete von $\alpha$}}{\text{Ankathete von $\alpha$}} \tag{1}
\end{align}
\begin{figure}[hb!]
  \centering
  \def\svgwidth{200px}
  \input{assets/main/rechtwinkliges_dreieck_tangens.path.pdf_tex}
  \caption{Rechtwinkliges Dreieck}
  \label{fig:rechtwinkliges_dreieck_tangens}
\end{figure}
\documentclass{standalone}
\begin{document}

\begin{frame}
  \begin{center}
    Einheitskreis
  \end{center}
  \section{Einheitskreis}
\end{frame}

\documentclass{standalone}

\begin{document}

\subsection{Beispiel}

\textbf{Aufgaben-Text:} Auf einem Koordinatensystem eines Radarschirms (\autoref{fig:radar}) wird die Lage von zwei Schiffen durch die Entfernung zum Hafen(0) und durch den Kurs gegen{"u}ber der x-Achse beschrieben. \\
\textbf{Aufgabe:} Ein Schiff \textbf{A} ist mit dem Kurs \textbf{30$^\circ$} gegen{"u}ber der x-Achse \textbf{einen Kilometer} weit gefahren. Welche Koordinaten im \textbf{x-y-Kooradinatensystem} hat es?\\
Welche Koordinaten hat das Schiff \textbf{B}, das mit dem Kurs \textbf{75$^\circ$} \textbf{einen Kilometer} weit gefahren ist?
\begin{figure}[hb!]
	\centering
	\def\svgwidth{250px}
	\input{tmp/Radar.path.svg.pdf_tex}
	\caption{Radar}
	\label{fig:radar}
\end{figure}
\\
\noindent
\textbf{L{"o}sung:}
\begin{wrapfigure}[9]{r}{0.5\textwidth}
	\def\svgwidth{250px}
	\input{tmp/radar_loesung.path.svg.pdf_tex}
	\caption{Radar L{"o}sung}
	\label{fig:radar_loesung}
\end{wrapfigure}
\\
Das Schiff \textbf{A} mit dem Kurs \textbf{30$^\circ$} befindet sich auf der x-Achse: etwa \textbf{0,86 Kilometer} und y-Achse: \textbf{0,5 Kilometer}. Also auf dem Punkt \textbf{A(0,86|0,5)} \\  \\
Das Schiff \textbf{B} mit dem Kurs \textbf{75$^\circ$} befindet sich auf der x-Achse: etwa \textbf{0,25 Kilometer} und y-Achse: \textbf{0,96 Kilometer}. Also auf dem Punkt \textbf{A(0,25|0,96)}

\end{document}


\documentclass{standalone}

\begin{document}

\subsection{Der Sinus und Kosinus am Einheitskreis}
\noindent
Dreiecke mit der \textbf{Hypotenusenl{"a}nge 1} kann man in einem Koordinatensystem auf folgenden Weise darstellen:
\begin{multicols}{2}
	\paragraph{1. }
	Die Endpunkte der \textbf{Hypotenuse} sind der Ursprung O und ein Punkt \textbf{P}, der auf einem Kreis O mit dem \textbf{Radius 1} liegt. Diesen Kreis nennt man den \textbf{Einheitskreis}.
	\paragraph{2. }
	Die Ecke mit dem rechten Winkel liegt auf der \textbf{x-Achse senkrecht unter P}. Der Punkt P hat somit Koordinaten \textbf{P(cos($\alpha$)|sin($\alpha$))}
\end{multicols}
\begin{figure}[hb!]
	\centering
	\def\svgwidth{250px}
	\input{tmp/sin_cos_einheitskreis.path.svg.pdf_tex}
	\caption{Sinus und Kosinus am Einheitskreis}
	\label{fig:sin_cos_einheitskreis}
\end{figure}

\end{document}


\documentclass{standalone}

\begin{document}

\begin{frame}
  \begin{center}
    Beziehungen zwischen Sinus, Kosinus und Tangens
  \end{center}
  \subsection{Beziehungen zwischen Sinus, Kosinus und Tangens}
\end{frame}

\begin{frame}{1}
	\onslide<1->{F{"u}r $0^\circ < \alpha < 90^\circ$ nimmt $\sin(\alpha)$ mit wachsendem $\alpha$ zu und $\cos(\alpha)$ ab(\hyperref[fig:0_alpha_90]{Abbildung \autoref{fig:0_alpha_90}}).}
  \onslide<2->{$\sin(0^\circ) = 0$, $\cos(0^\circ) = 1$ (\hyperref[fig:sin_null_cos_null_einheitskreis]{Abbildung \autoref{fig:sin_null_cos_null_einheitskreis}})}, \onslide<3->{$\sin(90^\circ) = 1$, $\cos(90^\circ) = 0$ (\hyperref[fig:sin_neunzig_cos_neunzig_einheitskreis]{Abbildung \autoref{fig:sin_neunzig_cos_neunzig_einheitskreis}}).}


	\only<1>{
		\begin{figure}[hb!]
			\centering
			\def\svgwidth{150px}
			\input{tmp/0_alpha_90.path.svg.pdf_tex}
			\caption{$0^\circ < \alpha < 90^\circ$}
			\label{fig:0_alpha_90}
		\end{figure}
	}

	\only<2>{
		\begin{figure}[hb!]
			\centering
			\def\svgwidth{150px}
			\input{tmp/sin_null_cos_null_einheitskreis.path.svg.pdf_tex}
			\caption{$\sin(0^\circ) = 0$, $\cos(0^\circ) = 1$}
			\label{fig:sin_null_cos_null_einheitskreis}
		\end{figure}
	}

	\only<3->{
		\begin{figure}[hb!]
			\centering
			\def\svgwidth{150px}
			\input{tmp/sin_neunzig_cos_neunzig_einheitskreis.path.svg.pdf_tex}
			\caption{$\sin(90^\circ) = 1$, $\cos(90^\circ) = 1$}
			\label{fig:sin_neunzig_cos_neunzig_einheitskreis}
		\end{figure}
	}
\end{frame}

\begin{frame}{2}

	Wendet man auf das im Einheitskreis dargestellte Dreieck den Satz des Pythagoras an(\hyperref[fig:einheitskreis_dreieck_pythagoras]{\autoref{fig:einheitskreis_dreieck_pythagoras}}), so erh{"a}lt man den f{"u}r jede Winkelweite g{"u}ltigen Zusammenhang \\
	$\sin^2(\alpha) + \cos^2(\alpha) = 1$. 
	\begin{figure}[hb!]
		\centering
		\def\svgwidth{150px}
		\input{tmp/einheitskreis_dreieck_pythagoras.path.svg.pdf_tex}
		\caption{Einheitskreis Dreieck Satz des Pythagoras}
		\label{fig:einheitskreis_dreieck_pythagoras}
	\end{figure}

\end{frame}

\begin{frame}{Beispiel}
	\begin{align}
		\onslide<1->{\sin^2(\alpha) + \cos^2(\alpha)                                                                    & = 1 \tag{1} \\}
		\onslide<2->{(\sin(45))^2 +} \onslide<3->{(\cos(45))^2                                                          & = 1 \tag{2} \\}
		\onslide<4->{\left(\frac{\sqrt{2}}{2}\right)^2 +} \onslide<5->{\left(\frac{\sqrt{2}}{2}\right)^2} \onslide<6->{ & = 1 \tag{3} \\}
		\onslide<7->{\frac{\sqrt{2^2}}{2^2} +} \onslide<8->{\frac{\sqrt{2^2}}{2^2}}                       \onslide<9->{ & = 1 \tag{4} \\}
		\onslide<10->{\frac{2}{4} + \frac{2}{4}                                                                         & = 1 \tag{5} \\}
		\onslide<11->{\frac{1}{2} + \frac{1}{2}                                                                         & = 1 \tag{6} \\}
		\onslide<12->{0,5 + 0,5                                                                                         & = 1 \tag{7} \\[-\baselineskip]\notag }
	\end{align}
\end{frame}

\begin{frame}{3}
	In \hyperref[fig:sin_neunzig_minus_alpha_cos_neunzig_minus_alpha]{Abbildung \autoref{fig:sin_neunzig_minus_alpha_cos_neunzig_minus_alpha}} sieht man: \\
	$\sin(90^\circ - \alpha) = x = \cos(\alpha)$ und $\cos(90^\circ - \alpha) = y = \sin(\alpha)$

	\begin{figure}[hb!]
		\centering
		\def\svgwidth{150px}
		\input{tmp/sin_neunzig_minus_alpha_cos_neunzig_minus_alpha.path.svg.pdf_tex}
		\caption{sin(90$^\circ$ - $\alpha$); cos(90$^\circ$ - $\alpha$)}
		\label{fig:sin_neunzig_minus_alpha_cos_neunzig_minus_alpha}
	\end{figure}
\end{frame}

\begin{frame}{Beispiel}
	\begin{align}
		\onslide<1->{\sin(90^\circ - \alpha)   & = x                   & = \cos(\alpha) \tag{1} \\}
		\onslide<2->{\sin(90^\circ - 30^\circ) & =} \onslide<3->{\frac{\sqrt{3}}{2} & =} \onslide<4->{\cos(30^\circ) \tag{2} \\[-\baselineskip]\notag}
	\end{align}

	\begin{figure}[hb!]
		\centering
		\def\svgwidth{150px}
		\input{tmp/sin_neunzig_minus_alpha_cos_neunzig_minus_alpha.path.svg.pdf_tex}
		\caption{sin(90$^\circ$ - $\alpha$); cos(90$^\circ$ - $\alpha$)}
		\label{fig:sin_neunzig_minus_alpha_cos_neunzig_minus_alpha_zwei}
	\end{figure}
\end{frame}

\begin{frame}{4}
	Ebenfalls in \hyperref[fig:sin_neunzig_minus_alpha_cos_neunzig_minus_alpha_drei]{Abbildung \autoref{fig:sin_neunzig_minus_alpha_cos_neunzig_minus_alpha_drei}}:
	\begin{align}
    \onslide<1->{\tan(\alpha) =} \onslide<2->{\frac{\onslide<2->{\text{Gegenkathete von $\alpha$}}}{\onslide<3->{\text{Ankathete von $\alpha$}}} =} \onslide<4->{\frac{y}{x} = \frac{\sin(\alpha)}{\cos(\alpha)}} \nonumber
	\end{align}
	
	\only<1-4>{
		\begin{figure}[hb!]
			\centering
			\def\svgwidth{150px}
			\input{tmp/sin_neunzig_minus_alpha_cos_neunzig_minus_alpha.path.svg.pdf_tex}
			\caption{sin(90$^\circ$ - $\alpha$); cos(90$^\circ$ - $\alpha$)}
			\label{fig:sin_neunzig_minus_alpha_cos_neunzig_minus_alpha_drei}
		\end{figure}
	}
	\only<5->{
		Wichtig:\\
		\begin{align}
			\onslide<5->{\tan(90) =} \onslide<6->{\frac{\sin(90)}{\cos(90)} =} \onslide<7->{\frac{1}{0} =} \onslide<8->{\text{\Lightning}} \nonumber
		\end{align}
			
    
	}
\end{frame}

\end{document}


\documentclass{standalone}
\begin{document}

\begin{frame}
  \begin{center}
    Einheitskreis - Definition
  \end{center}
  \subsection{Einheitskreis - Definition}
\end{frame}

\begin{frame}{Definition}
  Es gelten folgende Zusammenh{"a}nge:\\
  \begin{itemize}
    \item <2-> $\sin^2(\alpha) + \cos^2(\alpha) = 1$
    \item <3-> $\sin(90^\circ - \alpha) = \cos(\alpha)$ und $\cos(90^\circ - \alpha) = \sin(\alpha)$
    \item <4-> $\tan(\alpha) = \frac{\sin(\alpha)}{\cos(\alpha)}$, $\alpha \neq 90^\circ$
  \end{itemize}
\end{frame}

\end{document}


\documentclass{standalone}

\begin{document}

\subsection{Einheitskreis - Aufgabe}
\textbf{Aufgabe:}
$sin(\alpha) = 0,6$. \\

\noindent\textbf{Bestimme:}

\begin{center}
	\normalsize
	\begin{tabular}{ c c c c c }
		\normalsize\textbf{a)} $cos(\alpha)$            &

		\normalsize\textbf{b)} $tan(\alpha)$            &

		\normalsize\textbf{c)} $sin(90^\circ - \alpha)$ &

		\normalsize\textbf{d)} $cos(90^\circ - \alpha)$ &

		\normalsize\textbf{e)} $tan(90^\circ - \alpha)$
	\end{tabular}
\end{center}
\\

\noindent\textbf{a) L{"o}sung:}
\begin{align}
	sin^2(\alpha) + cos^2(\alpha) & = 1 \tag{1}                                    \\
	0,6^2 + cos^2(\alpha)         & = 1                         & |-0,6^2 \tag{2}  \\
	cos^2(\alpha)                 & = 1 - 0,36                  & |\sqrt{} \tag{3} \\
	cos(\alpha)                   & = \sqrt{1 - 0,36}   \tag{4}                    \\
	cos(\alpha)                   & = \sqrt{0,64}   \tag{5}                        \\
	cos(\alpha)                   & = 0,8   \tag{6}
\end{align}

\noindent\textbf{b) L{"o}sung:}
\begin{align}
	tan(\alpha) & = \frac{sin(\alpha)}{cos(\alpha)} \tag{1} \\
	tan(\alpha) & = \frac{0,6}{0,8} = \frac{6}{8} \tag{2}   \\
	tan(\alpha) & = \frac{3}{4} = 0,75 \tag{3}
\end{align}

\noindent\textbf{c) L{"o}sung:}
\begin{align}
	sin(90^\circ  - \alpha) & = cos(\alpha) = 0,8 \tag{1}
\end{align}

\noindent\textbf{d) L{"o}sung:}
\begin{align}
	cos(90^\circ  - \alpha) & = sin(\alpha) = 0,6 \tag{1}
\end{align}

\noindent\textbf{e) L{"o}sung:}
\begin{align}
	tan(90^\circ  - \alpha) & = \frac{sin(90^\circ  - \alpha)}{cos(90^\circ  - \alpha)} = \frac{0,8}{0,6} \tag{1} \\
	tan(90^\circ  - \alpha) & = \frac{8}{6} = \frac{4}{3} \tag{2}
\end{align}

\end{document}


\end{document}


\end{document}

