\documentclass[12pt,a4paper]{article}
\usepackage[utf8]{inputenc}
\usepackage[T1]{fontenc}
\usepackage[german]{babel} %deutsche Sprache
\usepackage{amsmath} %für Formeln
\usepackage{amsfonts} %Für Text in Formeln
\usepackage{amssymb} %Für mathematische Symbole
\usepackage{graphicx} %Für Bilder
\UseRawInputEncoding

\usepackage[absolute]{textpos}%Grafiken Positionieren

\usepackage{import} %für .pdf_tex Grafiken
\usepackage{color} %für .pdf_tex Grafiken
\usepackage{transparent} %für .pdf_tex Grafiken
\graphicspath{{tmp/}} %für .pdf_tex Grafiken
\usepackage{wrapfig} %Um Text neben Bildern zu positionieren

\usepackage{multicol} %Für mehr-spaltigen Text
\usepackage{xcolor,colortbl} %Für z.B Farben in Tabellen

\usepackage{hyperref} %Für Links wie z.B Abbildung X
\hypersetup{
	hidelinks
} %Um Link-Stile zu deaktivieren

\usepackage{adjustbox}%Um Elemente auf Seitenlänge zu bringen
\usepackage{standalone}%Um das Dokument aufzuteilen
\usepackage{ marvosym } %Füt lightning symbol
\usepackage{subcaption}
\captionsetup{compatibility=false}

\usepackage{scalerel,stackengine,amsmath} % Für ein "entspricht" Zeichen
\newcommand\equalhat{\mathrel{\stackon[1.5pt]{=}{\stretchto{%
    \scalerel*[\widthof{=}]{\wedge}{\rule{1ex}{3ex}}}{0.5ex}}}}

\usepackage[left=2cm,right=2cm,top=2cm,bottom=2cm]{geometry}  %Dokument Rand etc. einstellen
%\renewcommand{\familydefault}{\sfdefault} %sans-serif Schriftart
\title{Mit dem Sinus modellieren}
\author{Kirill Heitzler}
\date{29. April 2021}%\date{\today}

\begin{document}

\documentclass{standalone}
\begin{document}

\begin{frame}
  \begin{center}
    Zusammenfassung
  \end{center}
  \section{Zusammenfassung}
\end{frame}

\documentclass{standalone}
\begin{document}

\subsection{Einheitskreis}
\noindent Die Endpunkte eines Dreickecks mit der Hypotenusenl{"a}nge 1 bilden den Ursprung 0 und einen Punkt P, der auf einem Kreis um 0 mit dem Radius 1 liegt und den Einheitskreis bildet.
\vspace{-0.3cm}  %%%%% Vertical spacing
\begin{figure}[hb!]
  \center
  \def\svgwidth{150px}
  \input{tmp/0_alpha_90_360.path.svg.pdf_tex}
  \caption{Einheitskreis}
  \label{fig:0_alpha_90_360}
\end{figure}
\\\noindent Die Gegenkathete l{"a}sst sich mit $\sin(\alpha)$ und die Ankathete mit $\cos(\alpha)$ berechnen. 
\vspace{-0.3cm}  %%%%% Vertical spacing
\begin{figure}[hb!]
	\centering
	\def\svgwidth{150px}
	\input{tmp/sin_cos_einheitskreis.path.svg.pdf_tex}
	\caption{Sinus und Kosinus am Einheitskreis}
	\label{fig:sin_cos_einheitskreis}
\end{figure}
\end{document}
\documentclass{standalone}
\begin{document}

\subsection{Mit dem Sinus Modellieren}
Ordnet man jedem Winkel $\alpha$ mit $0^\circ \leq \alpha \leq 360^\circ$  seinen Sinuswert zu, so erh{"a}lt man die Sinusfunktion im Gradma{\ss} $f$ mit $f(\alpha) = sin(\alpha)$. Tr{"a}gt man die Werte der Sinusfunktion im Gradma{\ss} in ein entsprechendes Koordinatensystem erh{"a}lt man den Grafphen von $f$ (\autoref{fig:sinuswelle_two}).

\begin{figure}[hb!]
  \center
  \def\svgwidth{450px}
  \input{tmp/sinuswelle.path.svg.pdf_tex}
  \caption{$f(\alpha) = \sin(\alpha)$}
  \label{fig:sinuswelle_two}
\end{figure}

\end{document}

\end{document}

\end{document}

