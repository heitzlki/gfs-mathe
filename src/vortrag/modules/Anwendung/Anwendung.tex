\documentclass{standalone}
\begin{document}

\begin{frame}
  \begin{center}
    Anwendung
  \end{center}
  \section{Anwendung}
\end{frame}

\begin{frame}{Anwendung}
  \vspace{-\topsep}
  Auch wenn es uns nicht oft auff{"a}llt, viele technische Ger{"a}te bzw. Mechanismen verwenden die trigonometrischen Funktionen Sinus und Kosinus. Genauso wie viele mathematische Verfahren.\\
  \noindent Ein paar Beispiele:

  \begin{itemize}
    \setlength{\parskip}{0pt}
    \setlength{\itemsep}{0pt plus 1pt}
    \item<2-> Oszilloskop (elektronisches Messger{"a}t, das elektrische Spannungen in einen Verlaufsgraphen darstellt) 
    \item<3-> GPS - Global Positioning System (Positionsbestimmung mit Hilfe von Satelliten)
    \item<4-> Computergrafiken in 3D und 2D
    \item<5-> Landvermessungen
    \item<6-> Fourier Transformation (z. B. Anwendung beim Spektroskop f{"u}r Chemiker)
    \item<7-> Astronomen nutzten Spektroskope, um chemische Zusammensetzungen von weit entfernten Planeten zu bestimmen
  \end{itemize}
\end{frame}

\end{document}