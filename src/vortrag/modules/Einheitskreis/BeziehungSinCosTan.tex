\documentclass{standalone}

\begin{document}

\subsection{Beziehungen zwischen Sinus, Kosinus und Tangens}

\paragraph{1.}

\noindent F{"u}r $0^\circ < \alpha < 90^\circ$ nimmt $\sin(\alpha)$ mit wachsendem $\alpha$ zu und $\cos(\alpha)$ ab(\autoref{fig:0_alpha_90}). \\
$\sin(0^\circ) = 0$, $\cos(0^\circ) = 1$ (\autoref{fig:sin_null_cos_null_einheitskreis}), $\sin(90^\circ) = 1$, $\cos(90^\circ) = 0$ (\autoref{fig:sin_neunzig_cos_neunzig_einheitskreis}).

\newcommand{\smallImageSize}{130px} %Size for the Images saved in a variable
\begin{figure}[h!]
	\centering
	\begin{subfigure}[b]{0.3\linewidth}
		\def\svgwidth{\smallImageSize}
		\input{tmp/0_alpha_90.path.svg.pdf_tex}
		\caption{$0^\circ < \alpha < 90^\circ$}
		\label{fig:0_alpha_90}
	\end{subfigure}
	\begin{subfigure}[b]{0.3\linewidth}
		\def\svgwidth{\smallImageSize}
		\input{tmp/sin_null_cos_null_einheitskreis.path.svg.pdf_tex}
		\caption{$\sin(0^\circ) = 0$, $\cos(0^\circ) = 1$}
		\label{fig:sin_null_cos_null_einheitskreis}
	\end{subfigure}
	\begin{subfigure}[b]{0.3\linewidth}
		\def\svgwidth{\smallImageSize}
		\input{tmp/sin_neunzig_cos_neunzig_einheitskreis.path.svg.pdf_tex}
		\caption{$\sin(90^\circ) = 1$, $\cos(90^\circ) = 1$}
		\label{fig:sin_neunzig_cos_neunzig_einheitskreis}
	\end{subfigure}
	\caption{Beziehung 1}
	\label{fig:beziehung_eins}
\end{figure}

\newpage

\paragraph{2.}

Wendet man auf das im Einheitskreis dargestellte Dreieck den Satz des Pythagoras an(\autoref{fig:einheitskreis_dreieck_pythagoras}), so erh{"a}lt man den f{"u}r jede Winkelweite g{"u}ltigen Zusammenhang \\
$\sin^2(\alpha) + \cos^2(\alpha) = 1$. \\
Beispiel:

\begin{align}
	\sin^2(\alpha) + \cos^2(\alpha)                                       & = 1 \tag{1}  \\
	(\sin(45))^2 + (\cos(45))^2                                           & = 1  \tag{2} \\
	\left(\frac{\sqrt{2}}{2}\right)^2 + \left(\frac{\sqrt{2}}{2}\right)^2 & = 1 \tag{3}  \\
	\frac{\sqrt{2^2}}{2^2} + \frac{\sqrt{2^2}}{2^2}                       & = 1 \tag{4}  \\
	\frac{2}{4} + \frac{2}{4}                                             & = 1 \tag{5}  \\
	\frac{1}{2} + \frac{1}{2}                                             & = 1 \tag{6}  \\
	0,5 + 0,5                                                             & = 1 \tag{7}
\end{align}

\begin{figure}[hb!]
	\centering
	\def\svgwidth{200px}
	\input{tmp/einheitskreis_dreieck_pythagoras.path.svg.pdf_tex}
	\caption{Einheitskreis Dreieck Satz des Pythagoras}
	\label{fig:einheitskreis_dreieck_pythagoras}
\end{figure}

\paragraph{3.}

\begin{wrapfigure}[6]{r}{0.45\textwidth}
	\def\svgwidth{200px}
	\input{tmp/sin_neunzig_minus_alpha_cos_neunzig_minus_alpha.path.svg.pdf_tex}
	\caption{sin(90$^\circ$ - $\alpha$); cos(90$^\circ$ - $\alpha$)}
	\label{fig:sin_neunzig_minus_alpha_cos_neunzig_minus_alpha}
\end{wrapfigure}

In \autoref{fig:sin_neunzig_minus_alpha_cos_neunzig_minus_alpha} sieht man: \\
$\sin(90^\circ - \alpha) = x = \cos(\alpha)$ und \\
$\cos(90^\circ - \alpha) = y = \sin(\alpha)$ \\
\textbf{Beispiel:}
\begin{align}
	\sin(90^\circ - \alpha)   & = x                   & = \cos(\alpha) \tag{1}   \\
	\sin(90^\circ - 30^\circ) & =  \frac{\sqrt{3}}{2} & = \cos(30^\circ) \tag{2}
\end{align}

\paragraph{4.}

Ebenfalls in \autoref{fig:sin_neunzig_minus_alpha_cos_neunzig_minus_alpha}:\\
$tan(\alpha) = \frac{y}{x} = \frac{sin(\alpha)}{\cos(\alpha)}$.

\end{document}
