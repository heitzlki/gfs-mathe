\documentclass{standalone}

\begin{document}

\begin{frame}
  \begin{center}
    Beziehungen zwischen Sinus, Kosinus und Tangens
  \end{center}
  \subsection{Beziehungen zwischen Sinus, Kosinus und Tangens}
\end{frame}

\begin{frame}{1}
	\onslide<1->{F{"u}r $0^\circ < \alpha < 90^\circ$ nimmt $\sin(\alpha)$ mit wachsendem $\alpha$ zu und $\cos(\alpha)$ ab(\hyperref[fig:0_alpha_90]{Abbildung \autoref{fig:0_alpha_90}}).}
  \onslide<2->{$\sin(0^\circ) = 0$, $\cos(0^\circ) = 1$ (\hyperref[fig:sin_null_cos_null_einheitskreis]{Abbildung \autoref{fig:sin_null_cos_null_einheitskreis}})}, \onslide<3->{$\sin(90^\circ) = 1$, $\cos(90^\circ) = 0$ (\hyperref[fig:sin_neunzig_cos_neunzig_einheitskreis]{Abbildung \autoref{fig:sin_neunzig_cos_neunzig_einheitskreis}}).}


	\only<1>{
		\begin{figure}[hb!]
			\centering
			\def\svgwidth{150px}
			\input{tmp/0_alpha_90.path.svg.pdf_tex}
			\caption{$0^\circ < \alpha < 90^\circ$}
			\label{fig:0_alpha_90}
		\end{figure}
	}

	\only<2>{
		\begin{figure}[hb!]
			\centering
			\def\svgwidth{150px}
			\input{tmp/sin_null_cos_null_einheitskreis.path.svg.pdf_tex}
			\caption{$\sin(0^\circ) = 0$, $\cos(0^\circ) = 1$}
			\label{fig:sin_null_cos_null_einheitskreis}
		\end{figure}
	}

	\only<3->{
		\begin{figure}[hb!]
			\centering
			\def\svgwidth{150px}
			\input{tmp/sin_neunzig_cos_neunzig_einheitskreis.path.svg.pdf_tex}
			\caption{$\sin(90^\circ) = 1$, $\cos(90^\circ) = 1$}
			\label{fig:sin_neunzig_cos_neunzig_einheitskreis}
		\end{figure}
	}
\end{frame}

\begin{frame}{2}

	Wendet man auf das im Einheitskreis dargestellte Dreieck den Satz des Pythagoras an(\hyperref[fig:einheitskreis_dreieck_pythagoras]{\autoref{fig:einheitskreis_dreieck_pythagoras}}), so erh{"a}lt man den f{"u}r jede Winkelweite g{"u}ltigen Zusammenhang \\
	$\sin^2(\alpha) + \cos^2(\alpha) = 1$. 
	\begin{figure}[hb!]
		\centering
		\def\svgwidth{150px}
		\input{tmp/einheitskreis_dreieck_pythagoras.path.svg.pdf_tex}
		\caption{Einheitskreis Dreieck Satz des Pythagoras}
		\label{fig:einheitskreis_dreieck_pythagoras}
	\end{figure}

\end{frame}

\begin{frame}{Beispiel}
	\begin{align}
		\onslide<1->{\sin^2(\alpha) + \cos^2(\alpha)                                                                    & = 1 \tag{1} \\}
		\onslide<2->{(\sin(45))^2 +} \onslide<3->{(\cos(45))^2                                                          & = 1 \tag{2} \\}
		\onslide<4->{\left(\frac{\sqrt{2}}{2}\right)^2 +} \onslide<5->{\left(\frac{\sqrt{2}}{2}\right)^2} \onslide<6->{ & = 1 \tag{3} \\}
		\onslide<7->{\frac{\sqrt{2^2}}{2^2} +} \onslide<8->{\frac{\sqrt{2^2}}{2^2}}                       \onslide<9->{ & = 1 \tag{4} \\}
		\onslide<10->{\frac{2}{4} + \frac{2}{4}                                                                         & = 1 \tag{5} \\}
		\onslide<11->{\frac{1}{2} + \frac{1}{2}                                                                         & = 1 \tag{6} \\}
		\onslide<12->{0,5 + 0,5                                                                                         & = 1 \tag{7} \\[-\baselineskip]\notag }
	\end{align}
\end{frame}

\begin{frame}{3}
	In \hyperref[fig:sin_neunzig_minus_alpha_cos_neunzig_minus_alpha]{Abbildung \autoref{fig:sin_neunzig_minus_alpha_cos_neunzig_minus_alpha}} sieht man: \\
	$\sin(90^\circ - \alpha) = x = \cos(\alpha)$ und $\cos(90^\circ - \alpha) = y = \sin(\alpha)$

	\begin{figure}[hb!]
		\centering
		\def\svgwidth{150px}
		\input{tmp/sin_neunzig_minus_alpha_cos_neunzig_minus_alpha.path.svg.pdf_tex}
		\caption{sin(90$^\circ$ - $\alpha$); cos(90$^\circ$ - $\alpha$)}
		\label{fig:sin_neunzig_minus_alpha_cos_neunzig_minus_alpha}
	\end{figure}
\end{frame}

\begin{frame}{Beispiel}
	\begin{align}
		\onslide<1->{\sin(90^\circ - \alpha)   & = x                   & = \cos(\alpha) \tag{1} \\}
		\onslide<2->{\sin(90^\circ - 30^\circ) & =} \onslide<3->{\frac{\sqrt{3}}{2} & =} \onslide<4->{\cos(30^\circ) \tag{2} \\[-\baselineskip]\notag}
	\end{align}

	\begin{figure}[hb!]
		\centering
		\def\svgwidth{150px}
		\input{tmp/sin_neunzig_minus_alpha_cos_neunzig_minus_alpha.path.svg.pdf_tex}
		\caption{sin(90$^\circ$ - $\alpha$); cos(90$^\circ$ - $\alpha$)}
		\label{fig:sin_neunzig_minus_alpha_cos_neunzig_minus_alpha_zwei}
	\end{figure}
\end{frame}

\begin{frame}{4}
	Ebenfalls in \hyperref[fig:sin_neunzig_minus_alpha_cos_neunzig_minus_alpha_drei]{Abbildung \autoref{fig:sin_neunzig_minus_alpha_cos_neunzig_minus_alpha_drei}}:
	\begin{align}
    \onslide<1->{\tan(\alpha) =} \onslide<2->{\frac{\onslide<2->{\text{Gegenkathete von $\alpha$}}}{\onslide<3->{\text{Ankathete von $\alpha$}}} =} \onslide<4->{\frac{y}{x} = \frac{\sin(\alpha)}{\cos(\alpha)}} \nonumber
	\end{align}
	
	\only<1-4>{
		\begin{figure}[hb!]
			\centering
			\def\svgwidth{150px}
			\input{tmp/sin_neunzig_minus_alpha_cos_neunzig_minus_alpha.path.svg.pdf_tex}
			\caption{sin(90$^\circ$ - $\alpha$); cos(90$^\circ$ - $\alpha$)}
			\label{fig:sin_neunzig_minus_alpha_cos_neunzig_minus_alpha_drei}
		\end{figure}
	}
	\only<5->{
		Wichtig:\\
		\begin{align}
			\onslide<5->{\tan(90) =} \onslide<6->{\frac{\sin(90)}{\cos(90)} =} \onslide<7->{\frac{1}{0} =} \onslide<8->{\text{\Lightning}} \nonumber
		\end{align}
			
    
	}
\end{frame}

\end{document}
