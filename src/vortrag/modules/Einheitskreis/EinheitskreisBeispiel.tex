\documentclass{standalone}

\begin{document}

\subsection{Beispiel}

\textbf{Aufgaben-Text:} Auf einem Koordinatensystem eines Radarschirms (\autoref{fig:radar}) wird die Lage von zwei Schiffen durch die Entfernung zum Hafen(0) und durch den Kurs gegen{"u}ber der x-Achse beschrieben. \\
\textbf{Aufgabe:} Ein Schiff \textbf{A} ist mit dem Kurs \textbf{30$^\circ$} gegen{"u}ber der x-Achse \textbf{einen Kilometer} weit gefahren. Welche Koordinaten im \textbf{x-y-Kooradinatensystem} hat es?\\
Welche Koordinaten hat das Schiff \textbf{B}, das mit dem Kurs \textbf{75$^\circ$} \textbf{einen Kilometer} weit gefahren ist?
\begin{figure}[hb!]
	\centering
	\def\svgwidth{250px}
	\input{tmp/Radar.path.svg.pdf_tex}
	\caption{Radar}
	\label{fig:radar}
\end{figure}
\\
\noindent
\textbf{L{"o}sung:}
\begin{wrapfigure}[9]{r}{0.5\textwidth}
	\def\svgwidth{250px}
	\input{tmp/radar_loesung.path.svg.pdf_tex}
	\caption{Radar L{"o}sung}
	\label{fig:radar_loesung}
\end{wrapfigure}
\\
Das Schiff \textbf{A} mit dem Kurs \textbf{30$^\circ$} befindet sich auf der x-Achse: etwa \textbf{0,86 Kilometer} und y-Achse: \textbf{0,5 Kilometer}. Also auf dem Punkt \textbf{A(0,86|0,5)} \\  \\
Das Schiff \textbf{B} mit dem Kurs \textbf{75$^\circ$} befindet sich auf der x-Achse: etwa \textbf{0,25 Kilometer} und y-Achse: \textbf{0,96 Kilometer}. Also auf dem Punkt \textbf{A(0,25|0,96)}

\end{document}
