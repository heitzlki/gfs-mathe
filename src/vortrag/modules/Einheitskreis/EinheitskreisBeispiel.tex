\documentclass{standalone}

\begin{document}

\begin{frame}
  \begin{center}
    Einheitskreis - Beispiel
  \end{center}
  \subsection{Beispiel}
\end{frame}

\begin{frame}{Aufgaben-Text}
	Auf einem Koordinatensystem eines Radarschirms (\hyperref[fig:radar]{Abbildung \autoref{fig:radar}}) wird die Lage von zwei Schiffen durch die Entfernung zum Hafen(0) und durch den Kurs gegen{"u}ber der x-Achse beschrieben.
	\begin{figure}[hb!]
		\centering
		\def\svgwidth{150px}
		\input{tmp/radar.path.svg.pdf_tex}
		\caption{Radar}
		\label{fig:radar}
	\end{figure}
\end{frame}

\begin{frame}{Aufgaben A}
	Ein Schiff \textbf{A} ist mit dem Kurs \textbf{30$^\circ$} gegen{"u}ber der x-Achse \textbf{einen Kilometer} weit gefahren. Welche Koordinaten im \textbf{x-y-Koordinatensystem} hat es?
	\begin{figure}[hb!]
		\centering
		\def\svgwidth{150px}
		\input{tmp/radar_a.path.svg.pdf_tex}
		\caption{Radar}
		\label{fig:radar_a}
	\end{figure}
\end{frame}

\begin{frame}{L{"o}sung A}
	\only<1>{Sch{"a}tzungen?}
	\only<1>{
		\begin{figure}[hb!]
			\centering
			\def\svgwidth{150px}
			\input{tmp/radar_a.path.svg.pdf_tex}
			\caption{Radar L{"o}sung}
			\label{fig:radar_a_two}
		\end{figure}
	}

	\onslide<2->{Das Schiff \textbf{A} mit dem Kurs \textbf{30$^\circ$} befindet sich auf der x-Achse: etwa \textbf{0,86 Kilometer} und y-Achse: \textbf{0,5 Kilometer}. Also auf dem Punkt \textbf{A(0,86|0,5)}}
	\onslide<2->{
		\begin{figure}[hb!]
			\centering
			\def\svgwidth{150px}
			\input{tmp/radar_a_loesung.path.svg.pdf_tex}
			\caption{Radar L{"o}sung}
			\label{fig:radar_a_loesung}
		\end{figure}
	}
\end{frame}

\begin{frame}{Aufgaben B}
	Welche Koordinaten hat das Schiff \textbf{B}, das mit dem Kurs \textbf{75$^\circ$} \textbf{einen Kilometer} weit gefahren ist?
	\begin{figure}[hb!]
		\centering
		\def\svgwidth{150px}
		\input{tmp/radar_b.path.svg.pdf_tex}
		\caption{Radar}
		\label{fig:radar_b}
	\end{figure}
\end{frame}

\begin{frame}{L{"o}sung B}
	\only<1>{Sch{"a}tzungen?}
	\only<1>{
		\begin{figure}[hb!]
			\centering
			\def\svgwidth{150px}
			\input{tmp/radar_b.path.svg.pdf_tex}
			\caption{Radar L{"o}sung}
			\label{fig:radar_b_two}
		\end{figure}
	}

	\onslide<2->{Das Schiff \textbf{B} mit dem Kurs \textbf{75$^\circ$} befindet sich auf der x-Achse: etwa \textbf{0,25 Kilometer} und y-Achse: \textbf{0,96 Kilometer}. Also auf dem Punkt \textbf{A(0,25|0,96)}}
	\onslide<2->{
		\begin{figure}[hb!]
			\centering
			\def\svgwidth{150px}
			\input{tmp/radar_b_loesung.path.svg.pdf_tex}
			\caption{Radar L{"o}sung}
			\label{fig:radar_b_loesung}
		\end{figure}
	}
\end{frame}

\end{document}
