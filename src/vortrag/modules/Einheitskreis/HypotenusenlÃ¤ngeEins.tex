\documentclass{standalone}
\begin{document}

\begin{frame}
  \begin{center}
    Hypotenusenl{"a}nge 1
  \end{center}
  \subsection{Hypotenusenl{"a}nge 1}
\end{frame}

\begin{frame}{Hypotenusenl{"a}nge 1}
  \onslide<1->{Betr{"a}gt in einem rechtwinkligen Dreieck die Hypotenusenl{"a}nge in einer vorgegebenen L{"a}ngeneinheit 1, so lassen sich Sinus und Kosinus besonderst einfach darstellen.}
  \only<2-4>{
    \begin{align}
      \onslide<2->{\sin(\alpha) & = \dfrac{\text{Gegenkathete von $\alpha$}}{\text{Hypotenuse}}} \nonumber \\
      \onslide<3->{\sin(\alpha) & = \dfrac{\text{Gegenkathete von $\alpha$}}{\text{1}}} \nonumber \\
      \onslide<4->{\sin(\alpha) & = \text{Gegenkathete von $\alpha$} &= \sin(\alpha)} \nonumber \\[-\baselineskip]\nonumber
    \end{align}
  }

  \only<5-7>{
    \begin{align}
      \onslide<5->{\cos(\alpha) & = \dfrac{\text{Ankathete von $\alpha$}}{\text{Hypotenuse}}} \nonumber \\
      \onslide<6->{\cos(\alpha) & = \dfrac{\text{Ankathete von $\alpha$}}{\text{1}}} \nonumber \\
      \onslide<7->{\cos(\alpha) & = \text{Ankathete von $\alpha$} &= \cos(\alpha)} \nonumber \\[-\baselineskip]\nonumber
    \end{align}
  }

  \only<3>{
    \begin{figure}[hb!]
      \center
      \def\svgwidth{150px}
      \input{tmp/hypotenuse_1_gegenkathete.path.svg.pdf_tex}
      \caption{Hypotenusenl{"a}nge 1}
      \label{fig:hypotenuse_1_gegenkathete}
    \end{figure}
  }
  \only<4>{
    \begin{figure}[hb!]
      \center
      \def\svgwidth{150px}
      \input{tmp/hypotenuse_1_sin.path.svg.pdf_tex}
      \caption{Hypotenusenl{"a}nge 1}
      \label{fig:hypotenuse_1_sin}
    \end{figure}
  }

  \only<6>{
    \begin{figure}[hb!]
      \center
      \def\svgwidth{150px}
      \input{tmp/hypotenuse_1_ankathete.path.svg.pdf_tex}
      \caption{Hypotenusenl{"a}nge 1}
      \label{fig:hypotenuse_1_ankathete}
    \end{figure}
  }
  \only<7>{
    \begin{figure}[hb!]
      \center
      \def\svgwidth{150px}
      \input{tmp/hypotenuse_1_cos.path.svg.pdf_tex}
      \caption{Hypotenusenl{"a}nge 1}
      \label{fig:hypotenuse_1_cos}
    \end{figure}
  }
\end{frame}

\end{document}
