\documentclass{standalone}

\begin{document}

\begin{frame}
  \begin{center}
    Der Sinus und Kosinus am Einheitskreis
  \end{center}
  \subsection{Der Sinus und Kosinus am Einheitskreis}
\end{frame}

\begin{frame}{Dreieck mit Hypotenusenl{"a}nge 1}
	Dreiecke mit der \textbf{Hypotenusenl{"a}nge 1} kann man in einem Koordinatensystem auf folgenden Weise darstellen:
	\begin{figure}[hb!]
		\centering
		\def\svgwidth{200px}
		\input{tmp/rechtwinkliges_dreieck_hypotenuse_eins.path.svg.pdf_tex}
		\caption{Dreieck mit Hypotenusenl{"a}nge 1}
		\label{fig:rechtwinkliges_dreieck_hypotenuse_eins}
	\end{figure}
\end{frame}


\begin{frame}{1}
	Die Endpunkte der \textbf{Hypotenuse} sind der Ursprung 0 und ein Punkt \textbf{P}, der auf einem Kreis um 0 mit dem \textbf{Radius 1} liegt. Diesen Kreis nennt man den \textbf{Einheitskreis}.

	\begin{figure}[hb!]
		\centering
		\def\svgwidth{150px}
		\input{tmp/sin_cos_einheitskreis.path.svg.pdf_tex}
		\caption{Sinus und Kosinus am Einheitskreis}
		\label{fig:sin_cos_einheitskreis_1}
	\end{figure}
\end{frame}

\begin{frame}{2}
	Die Ecke mit dem rechten Winkel liegt auf der \textbf{x-Achse senkrecht unter P}. Der Punkt P hat somit die Koordinaten \textbf{P(cos($\alpha$)|sin($\alpha$))}

	\begin{figure}[hb!]
		\centering
		\def\svgwidth{150px}
		\input{tmp/sin_cos_einheitskreis.path.svg.pdf_tex}
		\caption{Sinus und Kosinus am Einheitskreis}
		\label{fig:sin_cos_einheitskreis_2}
	\end{figure}
\end{frame}

\end{document}
