\documentclass{standalone}
\begin{document}

\begin{frame}
  \begin{center}
    Mit dem Sinus Modellieren - Zusammenfassung
  \end{center}
  \subsection{Mit dem Sinus Modellieren - Zusammenfassung}
\end{frame}

\begin{frame}{Mit dem Sinus Modellieren - Zusammenfassung}
  \onslide<1->{Ordnet man jedem Winkel $\alpha$ mit $0^\circ \leq \alpha \leq 360^\circ$  seinen Sinuswert zu, so erh{"a}lt man die Sinusfunktion im Gradma{\ss} $f$ mit $f(\alpha) = sin(\alpha)$.}
  \onslide<2->{Tr{"a}gt man die Werte der Sinusfunktion im Gradma{\ss} in ein entsprechendes Koordinatensystem erh{"a}lt man den Grafphen von $f$ (\hyperref[fig:sinuswelle_two]{Abbildung \autoref{fig:sinuswelle_two}}).}
  \onslide<2->{
    \begin{figure}[hb!]
      \centering
      \def\svgwidth{300px}
      \input{tmp/sinuswelle.path.svg.pdf_tex}
      \caption{$f(\alpha) = \sin(\alpha)$}
      \label{fig:sinuswelle_two}
    \end{figure}
  }
\end{frame}

\end{document}