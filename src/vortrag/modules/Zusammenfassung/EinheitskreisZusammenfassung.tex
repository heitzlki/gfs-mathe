\documentclass{standalone}
\begin{document}

\begin{frame}
  \begin{center}
    Einheitskreis - Zusammenfassung
  \end{center}
  \subsection{Einheitskreis - Zusammenfassung}
\end{frame}

\begin{frame}{Einheitskreis - Zusammenfassung}
  Die Endpunkte eines Dreickecks mit der Hypotenusenl{"a}nge 1 bilden den Ursprung 0 und einen Punkt P, der auf einem Kreis um 0 mit dem Radius 1 liegt und den Einheitskreis bildet.
  \begin{figure}[hb!]
    \center
    \def\svgwidth{150px}
    \input{tmp/0_alpha_90_360.path.svg.pdf_tex}
    \caption{Einheitskreis}
    \label{fig:0_alpha_90_360_two}
  \end{figure}
\end{frame}

\begin{frame}{Einheitskreis - Zusammenfassung}
  Die Gegenkathete l{"a}sst sich mit $\sin(\alpha)$ und die Ankathete mit $\cos(\alpha)$ berechnen.
  \begin{figure}[hb!]
    \centering
    \def\svgwidth{150px}
    \input{tmp/sin_cos_einheitskreis.path.svg.pdf_tex}
    \caption{Sinus und Kosinus am Einheitskreis}
    \label{fig:sin_cos_einheitskreis_two}
  \end{figure}
\end{frame}

\end{document}