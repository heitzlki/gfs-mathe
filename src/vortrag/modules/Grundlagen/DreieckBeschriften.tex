\documentclass{standalone}

\begin{document}

\subsection{Rechtwinkliges Dreieck - Beschriftung}
\begin{figure}[hb!]
  \centering
  \def\svgwidth{200px}
  \input{tmp/rechtwinkliges_dreieck.path.svg.pdf_tex}
  \caption{Rechtwinkliges Dreieck}
  \label{fig:rechtwinkliges_dreieck}
\end{figure}
\noindent
Das Rechtwinklige Dreieck wird folgenderma{\ss}en wie in \autoref{fig:rechtwinkliges_dreieck} beschriftet. \\
Die Ecken werden mit den Buchstaben A, B, C gegen den Uhrzeigersinn bei A angefangen beschriftet. \\
Die Winkel $\alpha$, $\beta$, $\gamma$ werden in die Ecken der entsprechenden Buchstaben A, B, C gesetzt. \\
Die anliegende Kathete zu Winkel $\alpha$ wird "`Ankathete von $\alpha$"' genannt und die Kathete gegen{"u}ber von $\alpha$ wird "`Gegenkathete von $\alpha$"' genannt. \\
Die Hypotenuse liegt gegen{"u}ber des rechten Winkels $\gamma$.

\end{document}
