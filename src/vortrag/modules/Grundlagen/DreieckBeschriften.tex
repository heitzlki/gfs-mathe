\documentclass{standalone}

\begin{document}
\begin{frame}
  \begin{center}
    Rechtwinkliges Dreieck - Beschriftung
  \end{center}
  \subsection{Rechtwinkliges Dreieck - Beschriftung}
\end{frame}

\begin{frame}{Ecken}

  \only<1-5>{\begin{itemize}
    \item<2->Gegen den Uhrzeigersinn
    \item<3->A
    \item<4->B
    \item<5->C
  \end{itemize}}
  \noindent
  \only<6>{Die Ecken werden mit den Buchstaben A, B, C gegen den Uhrzeigersinn bei A angefangen beschriftet.}

  \only<1-2>{
    \begin{figure}[hb!]
      \centering
      \def\svgwidth{200px}
      \input{tmp/rechtwinkliges_dreieck.path.svg.pdf_tex}
      \caption{Rechtwinkliges Dreieck}
      \label{fig:rechtwinkliges_dreieck_ecken}
    \end{figure}
  }

  \only<3>{
    \begin{figure}[hb!]
      \centering
      \def\svgwidth{200px}
      \input{tmp/rechtwinkliges_dreieck.path.svg.pdf_tex}
      \caption{Rechtwinkliges Dreieck}
      \label{fig:rechtwinkliges_dreieck_ecke_A}
    \end{figure}
  }

  \only<4>{
    \begin{figure}[hb!]
      \centering
      \def\svgwidth{200px}
      \input{tmp/rechtwinkliges_dreieck.path.svg.pdf_tex}
      \caption{Rechtwinkliges Dreieck}
      \label{fig:rechtwinkliges_dreieck_ecke_B}
    \end{figure}
  }

  \only<5->{
    \begin{figure}[hb!]
      \centering
      \def\svgwidth{200px}
      \input{tmp/rechtwinkliges_dreieck.path.svg.pdf_tex}
      \caption{Rechtwinkliges Dreieck}
      \label{fig:rechtwinkliges_dreieck_ecke_C}
    \end{figure}
  }

\end{frame}

\begin{frame}{Winkel}
  \only<1-4>{\begin{itemize}
    \item<2->$\alpha$
    \item<3->$\beta$
    \item<4->$\gamma$
  \end{itemize}}
  \noindent
  \only<5>{Die Winkel $\alpha$, $\beta$, $\gamma$ werden in die Ecken der entsprechenden Buchstaben A, B, C gesetzt.}

  \only<1>{
    \begin{figure}[hb!]
      \centering
      \def\svgwidth{200px}
      \input{tmp/rechtwinkliges_dreieck.path.svg.pdf_tex}
      \caption{Rechtwinkliges Dreieck}
      \label{fig:rechtwinkliges_dreieck_winkel}
    \end{figure}
  }

  \only<2>{
    \begin{figure}[hb!]
      \centering
      \def\svgwidth{200px}
      \input{tmp/rechtwinkliges_dreieck.path.svg.pdf_tex}
      \caption{Rechtwinkliges Dreieck}
      \label{fig:rechtwinkliges_dreieck_ecken_winkel_alpha}
    \end{figure}
  }

  \only<3>{
    \begin{figure}[hb!]
      \centering
      \def\svgwidth{200px}
      \input{tmp/rechtwinkliges_dreieck.path.svg.pdf_tex}
      \caption{Rechtwinkliges Dreieck}
      \label{fig:rechtwinkliges_dreieck_ecken_winkel_beta}
    \end{figure}
  }

  \only<4->{
    \begin{figure}[hb!]
      \centering
      \def\svgwidth{200px}
      \input{tmp/rechtwinkliges_dreieck.path.svg.pdf_tex}
      \caption{Rechtwinkliges Dreieck}
      \label{fig:rechtwinkliges_dreieck_ecken_winkel_gamma}
    \end{figure}
  }
\end{frame}

\begin{frame}{Katheten}
    \only<1-3>{\begin{itemize}
      \item<2->"`Ankathete von $\alpha$"'
      \item<3->"`Gegenkathete von $\alpha$"'
    \end{itemize}}
    \noindent
    \only<4>{Die anliegende Kathete zu Winkel $\alpha$ wird "`Ankathete von $\alpha$"' genannt \pause und die Kathete gegen{"u}ber von $\alpha$ wird "`Gegenkathete von $\alpha$"' genannt.}
  
    \only<1>{
      \begin{figure}[hb!]
        \centering
        \def\svgwidth{200px}
        \input{tmp/rechtwinkliges_dreieck.path.svg.pdf_tex}
        \caption{Rechtwinkliges Dreieck}
        \label{fig:rechtwinkliges_dreieck_katheten}
      \end{figure}
    }
  
    \only<2>{
      \begin{figure}[hb!]
        \centering
        \def\svgwidth{200px}
        \input{tmp/rechtwinkliges_dreieck.path.svg.pdf_tex}
        \caption{Rechtwinkliges Dreieck}
        \label{fig:rechtwinkliges_dreieck_ankathete}
      \end{figure}
    }

    \only<3->{
      \begin{figure}[hb!]
        \centering
        \def\svgwidth{200px}
        \input{tmp/rechtwinkliges_dreieck.path.svg.pdf_tex}
        \caption{Rechtwinkliges Dreieck}
        \label{fig:rechtwinkliges_dreieck_gegenkathete}
      \end{figure}
    }
\end{frame}

\begin{frame}{Hypotenuse}
  \only<1-2>{\begin{itemize}
    \item<2->"`Hypotenuse"'
  \end{itemize}}
  \noindent
  \only<3>{Die Hypotenuse liegt gegen{"u}ber des rechten Winkels $\gamma$.}

  \only<1>{
    \begin{figure}[hb!]
      \centering
      \def\svgwidth{200px}
      \input{tmp/rechtwinkliges_dreieck.path.svg.pdf_tex}
      \caption{Rechtwinkliges Dreieck}
      \label{fig:rechtwinkliges_dreieck_hypotenuse_one}
    \end{figure}
  }

  \only<2->{
    \begin{figure}[hb!]
      \centering
      \def\svgwidth{200px}
      \input{tmp/rechtwinkliges_dreieck.path.svg.pdf_tex}
      \caption{Rechtwinkliges Dreieck}
      \label{fig:rechtwinkliges_dreieck_hypotenuse_two}
    \end{figure}
  }
\end{frame}

\end{document}
