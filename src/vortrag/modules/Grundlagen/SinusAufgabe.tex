\documentclass{standalone}

\begin{document}

\subsection{Sinus - Beispiel}
\textbf{Gegenkathete von $\alpha$ mithilfe des Sinus berechnen}: \\
\textbf{Aufgabe:} Berechne die H{"o}he des Freiburger M{"u}nsters. Das rechtwinklige Dreieck in \autoref{fig:rechtwinkliges_dreieck_am_muenster} besitzt einen rechten Winkel (90$^\circ$), die Hypotenuse 164,05 Meter und die Winkelweite des Winkels $\alpha$ mit 45$^\circ$. Berechne die Gegenkathete von $\alpha$ namens x.\\
\begin{figure}[hb!]
  \centeringf
  \centering
  \def\svgwidth{300px}
  \input{tmp/rechtwinkliges_dreieck_am_muenster.path.svg.pdf_tex}
  \caption{Rechtwinkliges Dreieck am M{"u}nster}
  \label{fig:rechtwinkliges_dreieck_am_muenster}
\end{figure}

\noindent\textbf{Rechnung:}
\begin{align}
  \sin(\alpha)                  & = \dfrac{\text{Gegenkathete von $\alpha$}}{\text{Hypotenuse}} \tag{1}                                                     \\
  \sin(45^\circ )               & = \dfrac{x}{164,05m}                                                                             & |\cdot 164,05m \tag{2} \\
  \sin(45^\circ ) \cdot 164,05m & = x                                                                                      \tag{3}                          \\
  x                             & \cong 116m
\end{align}
\textbf{Antwort:} Die Gegenkathete von $\alpha$ betr{"a}gt etwa 116 Meter, somit ist das M{"u}nster auch etwa 116 Meter gro{\ss}.

\end{document}