\documentclass{standalone}

\begin{document}


\begin{frame}
  \begin{center}
    Sinus - Beispiel\\
    Gegenkathete von $\alpha$ mithilfe des Sinus berechnen
  \end{center}
  \subsection{Sinus - Beispiel}
\end{frame}

\begin{frame}{Aufgabe}
  Berechne die H{"o}he des Freiburger M{"u}nsters. Das rechtwinklige Dreieck in \hyperref[fig:rechtwinkliges_dreieck_am_muenster]{Abbildung \autoref{fig:rechtwinkliges_dreieck_am_muenster}} besitzt einen rechten Winkel (90$^\circ$), die Hypotenuse 164,05 Meter und die Winkelweite des Winkels $\alpha$ mit 45$^\circ$. Berechne die Gegenkathete von $\alpha$ namens x.\\
  \begin{figure}[hb!]
    \centeringf
    \centering
    \def\svgwidth{300px}
    \input{tmp/rechtwinkliges_dreieck_am_muenster.path.svg.pdf_tex}
    \caption{Rechtwinkliges Dreieck am M{"u}nster}
    \label{fig:rechtwinkliges_dreieck_am_muenster}
  \end{figure}
\end{frame}

\begin{frame}{Rechnung}
  \begin{itemize}
    \item<1-> $\alpha = 45^\circ$
    \item<2-> $\text{Hypotenuse} = 164,05m$
    \item<3-> Gegenkathete von $\alpha = x$
  \end{itemize}
  
  \begin{align}
    \onslide<4->{\sin(\alpha)                  & = \dfrac{\text{Gegenkathete von $\alpha$}}{\text{Hypotenuse}} \tag{1} \\}
    \onslide<5->{\sin(45^\circ )               & =} \onslide<6->{\dfrac{\onslide<6->{x}}{\onslide<7->{164,05m}}} & \onslide<8->{|\cdot 164,05m \tag{2} \\}
    \onslide<9->{\sin(45^\circ ) \cdot 164,05m & =} \onslide<10->{x \tag{3} \\}
    \onslide<11->{x                            & \cong} \onslide<12->{116m \tag{4} \\ \nonumber }
  \end{align}
\end{frame}

\begin{frame}{Antwort}
  \onslide<2->{Die Gegenkathete von $\alpha$ betr{"a}gt etwa 116 Meter, somit ist das M{"u}nster auch etwa 116 Meter gro{\ss}.}
  \onslide<1->{
    \begin{figure}[hb!]
      \centering
      \def\svgwidth{300px}
      \input{tmp/rechtwinkliges_dreieck_am_muenster.path.svg.pdf_tex}
      \caption{Rechtwinkliges Dreieck am M{"u}nster}
      \label{fig:rechtwinkliges_dreieck_am_muenster_loesung}
    \end{figure}
  }
\end{frame}

\end{document}