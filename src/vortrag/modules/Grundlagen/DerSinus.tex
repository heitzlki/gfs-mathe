\documentclass{standalone}

\begin{document}

\subsection{Der Sinus}

\textbf{Definition:} In einem rechtwinkligen Dreieck (\autoref{fig:rechtwinkliges_dreieck_sinus_def}) nennt man zu einem Winkel $\alpha$ des Dreiecks das Streckenverh{"a}ltnis
\begin{align}
  \sin(\alpha) & = \dfrac{\text{Gegenkathete von $\alpha$}}{\text{Hypotenuse}} \nonumber
\end{align}
den \textbf{Sinus von $\alpha$}.
\begin{figure}[hb!]
  \centering
  \def\svgwidth{200px}
  \input{tmp/rechtwinkliges_dreieck_sinus.path.svg.pdf_tex}
  \caption{Rechtwinkliges Dreieck}
  \label{fig:rechtwinkliges_dreieck_sinus_def}
\end{figure}

\end{document}
