\documentclass{standalone}

\begin{document}
\begin{frame}
  \begin{center}
    Der Sinus
  \end{center}
  \subsection{Der Sinus}
\end{frame}

\begin{frame}{Definition}

  \onslide<1->{In einem rechtwinkligen Dreieck (\hyperref[fig:rechtwinkliges_dreieck_sinus_def]{Abbildung \autoref{fig:rechtwinkliges_dreieck_sinus_def}}) nennt man zu einem Winkel $\alpha$ des Dreiecks das Streckenverh{"a}ltnis}

  \begin{align}
    \onslide<2->{\sin(\alpha) & =} \onslide<3->{\dfrac{\onslide<3->{\text{Gegenkathete von $\alpha$}}}{\onslide<4->{\text{Hypotenuse}}}} \nonumber
  \end{align}

  \onslide<5->{\noindent den \textbf{Sinus von $\alpha$}.}

  \only<1-2>{
    \begin{figure}[hb!]
      \centering
      \def\svgwidth{200px}
      \input{tmp/rechtwinkliges_dreieck_sinus.path.svg.pdf_tex}
      \caption{Rechtwinkliges Dreieck}
      \label{fig:rechtwinkliges_dreieck_sinus_def}
    \end{figure}
  }

  \only<3>{
    \begin{figure}[hb!]
      \centering
      \def\svgwidth{200px}
      \input{tmp/rechtwinkliges_dreieck_sinus.path.svg.pdf_tex}
      \caption{Rechtwinkliges Dreieck}
      \label{fig:rechtwinkliges_dreieck_sinus_def_gegenkathete}
    \end{figure}
  }

  \only<4->{
    \begin{figure}[hb!]
      \centering
      \def\svgwidth{200px}
      \input{tmp/rechtwinkliges_dreieck_sinus.path.svg.pdf_tex}
      \caption{Rechtwinkliges Dreieck}
      \label{fig:rechtwinkliges_dreieck_sinus_def_hypotenuse}
    \end{figure}
  }
\end{frame}

\end{document}