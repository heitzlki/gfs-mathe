\documentclass{standalone}

\begin{document}

\subsection{Der Kosinus und der Tangens}
\textbf{Sinus von $\alpha$:}
\begin{align}
  \sin(\alpha) & = \dfrac{\text{Gegenkathete von $\alpha$}}{\text{Hypotenuse}} \nonumber
\end{align}
\begin{figure}[hb!]
  \centering
  \def\svgwidth{200px}
  \input{tmp/rechtwinkliges_dreieck_sinus.path.svg.pdf_tex}
  \caption{Rechtwinkliges Dreieck}
  \label{fig:rechtwinkliges_dreieck_sinus}
\end{figure}
\\
\textbf{Cosinus von $\alpha$:}
\begin{align}
  \cos(\alpha) & = \dfrac{\text{Ankathete von $\alpha$}}{\text{Hypotenuse}} \nonumber
\end{align}
\begin{figure}[hb!]
  \centering
  \def\svgwidth{200px}
  \input{tmp/rechtwinkliges_dreieck_cosinus.path.svg.pdf_tex}
  \caption{Rechtwinkliges Dreieck}
  \label{fig:rechtwinkliges_dreieck_cosinus}
\end{figure}
\\
\textbf{Tangens von $\alpha$:}
\begin{align}
  \tan(\alpha) & = \dfrac{\text{Gegenkathete von $\alpha$}}{\text{Ankathete von $\alpha$}} \nonumber
\end{align}
\begin{figure}[hb!]
  \centering
  \def\svgwidth{200px}
  \input{tmp/rechtwinkliges_dreieck_tangens.path.svg.pdf_tex}
  \caption{Rechtwinkliges Dreieck}
  \label{fig:rechtwinkliges_dreieck_tangens}
\end{figure}

\end{document}
