\documentclass{standalone}

\begin{document}

\begin{frame}
  \begin{center}
    Der Kosinus und der Tangens
  \end{center}
  \subsection{Der Kosinus und der Tangens}
\end{frame}

\begin{frame}{Sinus von $\alpha$}
  \begin{align}
    \onslide<1->{\sin(\alpha) & =} \onslide<2->{\dfrac{\onslide<2->{\text{Gegenkathete von $\alpha$}}}{\onslide<3->{\text{Hypotenuse}}}} \nonumber
  \end{align}

  \only<1>{
    \begin{figure}[hb!]
      \centering
      \def\svgwidth{200px}
      \input{tmp/rechtwinkliges_dreieck_sinus_def.path.svg.pdf_tex}
      \caption{Rechtwinkliges Dreieck}
      \label{fig:rechtwinkliges_dreieck_sinus_def_two}
    \end{figure}
  }

  \only<2>{
    \begin{figure}[hb!]
      \centering
      \def\svgwidth{200px}
      \input{tmp/rechtwinkliges_dreieck_sinus_def_gegenkathete.path.svg.pdf_tex}
      \caption{Rechtwinkliges Dreieck}
      \label{fig:rechtwinkliges_dreieck_sinus_def_gegenkathete}
    \end{figure}
  }

  \only<3->{
    \begin{figure}[hb!]
      \centering
      \def\svgwidth{200px}
      \input{tmp/rechtwinkliges_dreieck_sinus.path.svg.pdf_tex}
      \caption{Rechtwinkliges Dreieck}
      \label{fig:rechtwinkliges_dreieck_sinus}
    \end{figure}
  }
\end{frame}

\begin{frame}{Cosinus von $\alpha$}
  \begin{align}
    \onslide<1->{\cos(\alpha) & =} \onslide<2->{\dfrac{\onslide<2->{\text{Ankathete von $\alpha$}}}{\onslide<3->{\text{Hypotenuse}}}} \nonumber
  \end{align}
  
  \only<1>{
    \begin{figure}[hb!]
      \centering
      \def\svgwidth{200px}
      \input{tmp/rechtwinkliges_dreieck_sinus_def.path.svg.pdf_tex}
      \caption{Rechtwinkliges Dreieck}
      \label{fig:rechtwinkliges_dreieck_cosinus_def}
    \end{figure}
  }

  \only<2>{
    \begin{figure}[hb!]
      \centering
      \def\svgwidth{200px}
      \input{tmp/rechtwinkliges_dreieck_cosinus_ankathete.path.svg.pdf_tex}
      \caption{Rechtwinkliges Dreieck}
      \label{fig:rechtwinkliges_dreieck_cosinus_ankathete}
    \end{figure}
  }

  \only<3->{
    \begin{figure}[hb!]
      \centering
      \def\svgwidth{200px}
      \input{tmp/rechtwinkliges_dreieck_cosinus.path.svg.pdf_tex}
      \caption{Rechtwinkliges Dreieck}
      \label{fig:rechtwinkliges_dreieck_cosinus}
    \end{figure}
  }
\end{frame}

\begin{frame}{Tangens von $\alpha$}
  \begin{align}
    \onslide<1->{\tan(\alpha) & =} \onslide<2->{\dfrac{\onslide<2->{\text{Gegenkathete von $\alpha$}}}{\onslide<3->{\text{Ankathete von $\alpha$}}}} \nonumber
  \end{align}
  
  \only<1>{
    \begin{figure}[hb!]
      \centering
      \def\svgwidth{200px}
      \input{tmp/rechtwinkliges_dreieck_sinus_def.path.svg.pdf_tex}
      \caption{Rechtwinkliges Dreieck}
      \label{fig:rechtwinkliges_dreieck_tangens_def}
    \end{figure}
  }

  \only<2>{
    \begin{figure}[hb!]
      \centering
      \def\svgwidth{200px}
      \input{tmp/rechtwinkliges_dreieck_tangens_gegenkathete.path.svg.pdf_tex}
      \caption{Rechtwinkliges Dreieck}
      \label{fig:rechtwinkliges_dreieck_tangens_gegenkathete}
    \end{figure}
  }

  \only<3->{
    \begin{figure}[hb!]
      \centering
      \def\svgwidth{200px}
      \input{tmp/rechtwinkliges_dreieck_tangens.path.svg.pdf_tex}
      \caption{Rechtwinkliges Dreieck}
      \label{fig:rechtwinkliges_dreieck_tangens}
    \end{figure}
  }
\end{frame}

\end{document}
