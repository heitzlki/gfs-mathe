\documentclass{standalone}
\begin{document}

\subsection{Graph einer Sinusfunktion zeichnen}

\definecolor{MyGrey}{rgb}{0.8,0.8,0.8}
\begin{adjustbox}{width=\textwidth}
  \begin{tabular}{ |>{\columncolor{MyGrey}}c|c|c|c|c|c|c|c|c|c|c|c|c|c| }
    \hline
    \rowcolor{MyGrey}
    Zeit t (in s)   & 0         & 5          & 10         & 15         & 20          & 25          & 30          & 35          & 40          & 45          & 50          & 55          & 60          \\
    \hline
    Winkel $\alpha$ & 0$^\circ$ & 30$^\circ$ & 60$^\circ$ & 90$^\circ$ & 120$^\circ$ & 150$^\circ$ & 180$^\circ$ & 210$^\circ$ & 240$^\circ$ & 270^$\circ$ & 300$^\circ$ & 330$^\circ$ & 360$^\circ$ \\
    \hline
    Höhe h (in m)   & 0         & 0,5        & 0,87       & 1          & 0,87        & 0,5         & 0           & -0,5        & -0,87       & -1          & -0,87       & -0,5        & 0           \\
    \hline
  \end{tabular}
\end{adjustbox}

\begin{figure}[hb!]
  \center
  \def\svgwidth{500px}
  \input{tmp/sinuswelle_zeichnen.path.svg.pdf_tex}
  \caption{Sinuswelle Zeichnen}
  \label{fig:sinuswelle_zeichnen}
\end{figure}

\end{document}