\documentclass{standalone}
\begin{document}

\begin{frame}
  \begin{center}
    Funktion f mit f($\alpha$)
  \end{center}
  \subsection{Funktion f mit f($\alpha$)}
\end{frame}

\begin{frame}{Funktion f mit f($\alpha$)}
  \onslide<1->{Ordnet man jedem Winkel $\alpha$ mit 0$^\circ$  $\leq \alpha \leq$ 360$^\circ$  seinen Sinuswert zu, so erh{"a}lt man eine Funktion f mit f($\alpha$) = sin($\alpha$).\\}
  \onslide<2->{Man kann mithilfe des Graphen von f (\hyperref[fig:sinuswelle]{Abbildung \autoref{fig:sinuswelle}}) zu gegebenem Winkel den Sinuswert n{"a}herungsweise ablesen oder n{"a}herungsweise Winkel mit vorgegebenem Sinuswert ermitteln.}

  \onslide<2->{
    \begin{figure}[hb!]
      \center
      \def\svgwidth{300px}
      \input{tmp/sinuswelle.path.svg.pdf_tex}
      \caption{$f(\alpha) = \sin(\alpha)$}
      \label{fig:sinuswelle}
    \end{figure}
    }
\end{frame}

\end{document}