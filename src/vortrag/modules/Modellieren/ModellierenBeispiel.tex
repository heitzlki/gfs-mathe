\documentclass{standalone}
\begin{document}

\begin{frame}
  \begin{center}
    Mit dem Sinus modellieren - Beispiel
  \end{center}
  \subsection{Beispiel}
\end{frame}

\begin{frame}{Aufgabe}
  \onslide<1->{Bei einem Shaufelraddampfer dreht sich das Rad mit dem Durchmesser 2 Meter einmal vollst{"a}ndig in 60 Sekunden(\hyperref[fig:schaufelraddamper]{Abbildung \autoref{fig:schaufelraddamper}}). In welcher H{"o}he {"u}ber dem Wasserspiegel liegt der rot markierte Punkt A?\\
  Erstelle eine Wertetabelle in 5 Sekunden-Schritten.}

  \onslide<2->{
    \begin{figure}[hb!]
      \center
      \def\svgwidth{150px}
      \input{tmp/schaufelraddamper.path.svg.pdf_tex}
      \caption{Schaufelraddampfer}
      \label{fig:schaufelraddamper}
    \end{figure}
  }
  
  \definecolor{MyGrey}{rgb}{0.8,0.8,0.8}
  \onslide<3->{
    \begin{center}
      \begin{adjustbox}{width=0.8\textwidth}
        \begin{tabular}{ |>{\columncolor{MyGrey}}c|c|c|c|c|c|c|c|c|c|c|c|c|c| }
          \hline
          \rowcolor{MyGrey}
          Zeit t (in s)    & 0         & 5          & 10         & ... & 60          \\
          \hline
          Winkel $\alpha$  & 0$^\circ$ & 30$^\circ$ & 60$^\circ$ &     & 360$^\circ$ \\
          \hline
          H{"o}he h (in m) & 0         & 0,5        & 0,87       &     & 0           \\
          \hline
        \end{tabular}
      \end{adjustbox}
    \end{center}
  }
\end{frame}



% \noindent\textbf{L{"o}sung:}\\

% \definecolor{MyGrey}{rgb}{0.8,0.8,0.8}
% \begin{adjustbox}{width=\textwidth}
%   \begin{tabular}{ |>{\columncolor{MyGrey}}c|c|c|c|c|c|c|c|c|c|c|c|c|c| }
%     \hline
%     \rowcolor{MyGrey}
%     Zeit t (in s)    & 0         & 5          & 10         & 15         & 20          & 25          & 30          & 35          & 40          & 45          & 50          & 55          & 60          \\
%     \hline
%     Winkel $\alpha$  & 0$^\circ$ & 30$^\circ$ & 60$^\circ$ & 90$^\circ$ & 120$^\circ$ & 150$^\circ$ & 180$^\circ$ & 210$^\circ$ & 240$^\circ$ & 270$^\circ$ & 300$^\circ$ & 330$^\circ$ & 360$^\circ$ \\
%     \hline
%     H{"o}he h (in m) & 0         & 0,5        & 0,87       & 1          & 0,87        & 0,5         & 0           & -0,5        & -0,87       & -1          & -0,87       & -0,5        & 0           \\
%     \hline
%   \end{tabular}
% \end{adjustbox}

\end{document}