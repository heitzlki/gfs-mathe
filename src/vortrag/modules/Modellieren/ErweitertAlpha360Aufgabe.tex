\documentclass{standalone}
\begin{document}

\begin{frame}
  \begin{center}
    Erweiterter Winkel $\alpha$ mit 90$^\circ$  $<$ $\alpha$ $\leq$ 360$^\circ$  - Aufgabe
  \end{center}
  \subsection{Erweiterter Winkel $\alpha$ mit 90$^\circ$  $<$ $\alpha$ $\leq$ 360$^\circ$  - Aufgabe}
\end{frame}

\begin{frame}{Aufgabe}
  \onslide<1->{Ein Punkt P bewegt sich auf dem Einheitskreis(\hyperref[fig:alpha_0]{Abbildung \autoref{fig:alpha_0}}) gegen den Uhrzeigersinn. F{"u}r $\alpha$ = 0$^\circ$  befindet er sich im Punkt(1|0).}

  \onslide<2->{Bestimme}
  \onslide<3->{
    \begin{itemize}
      \item<3-> a) Gib die x- und y-Koordinaten des Punktes P f{"u}r $\alpha$ = 140$^\circ$  und f{"u}r $\alpha$ = 310$^\circ$ an.
      \item<4-> b) Bestimme zwei verschiedene Werte f{"u}r $\alpha$, sodass seine y-Koordinate 0,8 betr{"a}gt.
    \end{itemize}
  }
  
  \onslide<1->{
    \begin{figure}[hb!]
      \centering
      \def\svgwidth{100px}
      \input{tmp/alpha_0.path.svg.pdf_tex}
      \caption{$\alpha = 0^\circ $}
      \label{fig:alpha_0}
    \end{figure} 
  }
\end{frame}

\begin{frame}{a) L{"o}sung}\\
  \onslide<1->{F{"u}r $\alpha$ = 140$^\circ$}\onslide<4->{: Punkt (\onslide<7->{-0,77}|\onslide<4->{0,64})}
  \begin{align}
    \onslide<2->{\sin(\alpha)     & = y \tag{1}           \\}
    \onslide<3->{\sin(140^\circ ) & \approx} \onslide<4->{0,64 \tag{2}  \\}
    \onslide<5->{\cos(\alpha)     & = x \tag{3}           \\}
    \onslide<6->{\cos(140^\circ ) & \approx} \onslide<7->{-0,77 \tag{4} \\[-\baselineskip]\notag}
  \end{align}

  \onslide<8->{F{"u}r $\alpha$ = 310$^\circ$}\onslide<11->{: Punkt (\onslide<14->{0,64}|\onslide<11->{-0,77})}
  \begin{align}
    \onslide<9->{\sin(\alpha)     & = y \tag{1}           \\}
    \onslide<10->{\sin(310^\circ ) & \approx} \onslide<11->{-0,77 \tag{2} \\}
    \onslide<12->{\cos(\alpha)     & = x \tag{3}           \\}
    \onslide<13->{\cos(310^\circ ) & \approx} \onslide<14->{0,64 \tag{4} \\[-\baselineskip]\notag}
  \end{align}

\end{frame}

% \onslide<1->{F{"u}r $\alpha$ = 310$^\circ$}: Punkt (\onslide<7->{0,64}|\onslide<4->{-0,77})
% \begin{align}
%   \onslide<2->{\sin(\alpha)     & = y \tag{1}           \\}
%   \onslide<3->{\sin(310^\circ ) & \approx} \onslide<4->{-0,77 \tag{2} \\}
%   \onslide<5->{\cos(\alpha)     & = x \tag{3}           \\}
%   \onslide<6->{\cos(310^\circ ) & \approx} \onslide<7->{0,64 \tag{4} \\[-\baselineskip]\notag}
% \end{align}

\begin{frame}{b) L{"o}sung}\\
  \onslide<1->{F{"u}r $\alpha_1$}\onslide<4->{: $sin(53,1^\circ) \approx 0,8$}
  \begin{align}
    \onslide<2->{\sin^-^1(y) & = \alpha \tag{1}  \\}
    \onslide<3->{\sin^-^1(0,8) & \approx} \onslide<4->{53,1^\circ \tag{2}  \\[-\baselineskip]\notag}
  \end{align}

  \onslide<5->{F{"u}r $\alpha_2$}\onslide<8->{: $\sin(126,9^\circ) \approx 0,8$}
  \begin{align}
    \onslide<6->{\sin^-^1(y) & = \alpha \tag{1}  \\}
    \onslide<7->{53,1^\circ - 180^\circ & =} \onslide<8->{126,9^\circ \tag{2}  \\[-\baselineskip]\notag}
  \end{align}
\end{frame}

\end{document}