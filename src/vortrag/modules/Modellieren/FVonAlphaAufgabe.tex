\documentclass{standalone}
\begin{document}

\begin{frame}
  \begin{center}
    Einen Zeitlichen Vorgang modellieren
  \end{center}
  \subsection{Einen Zeitlichen Vorgang modellieren}
\end{frame}

\begin{frame}{Einen Zeitlichen Vorgang modellieren}
  \noindent In einem Hafenbecken schwankt der Wasserstand periodisch um seinen Durchschnittswert (\hyperref[fig:wasserstand]{Abbildung \autoref{fig:wasserstand}})

  \begin{figure}[hb!]
    \center
    \def\svgwidth{300px}
    \input{tmp/wasserstand.path.svg.pdf_tex}
    \caption{Wasserstand}
    \label{fig:wasserstand}
  \end{figure}
\end{frame}


\begin{frame}{Aufgabe}
  \begin{itemize}
    \item<2-> a) Erl{"a}utere, wie man zu einem gegebenenen Zeitpunkt t die Winkelweite $\alpha$ erh{"a}lt und umgekehrt. Bestimme f{"u}r t = 5 (t in h) den zugeh{"o}rigen Winkel.
    \item<3-> b) Mit welcher Funktion kann man zu einem gegebenen Winkel $\alpha$ den Wasserstand berechnen? Berechne den Wasserstand 5 Stunden nach Beobachtungsbeginn.
  \end{itemize}
\end{frame}


% \begin{frame}{a) L{"o}sung}
%   \onslide<1->{Erl{"a}utere, wie man zu einem gegebenenen Zeitpunkt t die Winkelweite $\alpha$ erh{"a}lt und umgekehrt. Bestimme f{"u}r t = 5 (t in h) den zugeh{"o}rigen Winkel.}

%   \begin{align}
%     \onslide<1->{12h & \equalhat 360^\circ        & |:12\tag{1} \\}
%     \onslide<2->{1h  & \equalhat 30^\circ \tag{2}}
%   \end{align}

%   \noindent 12h in \hyperref[fig:wasserstand]{Abbildung \autoref{fig:wasserstand}} entsprechen 360$^\circ$, also entspricht 1h dem Winkel 30$^\circ$.\\
%   Daraus Folgt $\alpha = t \cdot 30^\circ$ und $t = \frac{\alpha}{30^\circ}$ (t in h). F{"u}r t = 5 erh{"a}lt man $\alpha = 5 \cdot 30^\circ = 150^\circ$

%   \begin{align}
%     \alpha & = t \cdot 30^\circ     \nonumber      \\
%     t      & = \frac{\alpha}{30^\circ}   \nonumber
%   \end{align}
%   \begin{align}
%     \alpha & = 5 \cdot 30^\circ \tag{1} \\ & = 150^\circ \tag{2}
%   \end{align}
% \end{frame}

% \noindent \textbf{b) L{"o}sung}\\
% Da der Wasserstand zwischen -0,2 und 0,2 um den Durchschnittswert pendelt (\autoref{fig:wasserstand}), gilt:
% \begin{align}
%   f(\alpha) & = 0,2 \cdot sin(\alpha)     \nonumber
% \end{align}
% \noindent F{"u}r t = 5:
% \begin{align}
%   \alpha       & = 5 \cdot 30^\circ     \tag{1}      \\
%                & = 150^\circ   \tag{2}               \\
%   f(150^\circ) & = 0,2 \cdot \sin(150^\circ) \tag{3} \\
%                & = 0,1 \tag{4}
% \end{align}\\
% \noindent Nach 5 Stunden liegt der Wasserstand 10cm {"u}ber dem Durchschnittswert.


\end{document}