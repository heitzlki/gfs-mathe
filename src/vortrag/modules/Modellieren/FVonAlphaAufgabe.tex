\documentclass{standalone}
\begin{document}

\begin{frame}
  \begin{center}
    Einen Zeitlichen Vorgang modellieren
  \end{center}
  \subsection{Einen Zeitlichen Vorgang modellieren}
\end{frame}

\begin{frame}{Einen Zeitlichen Vorgang modellieren}
  \noindent In einem Hafenbecken schwankt der Wasserstand periodisch um seinen Durchschnittswert (\hyperref[fig:wasserstand]{Abbildung \autoref{fig:wasserstand}})

  \begin{figure}[hb!]
    \center
    \def\svgwidth{300px}
    \input{tmp/wasserstand.path.svg.pdf_tex}
    \caption{Wasserstand}
    \label{fig:wasserstand}
  \end{figure}
\end{frame}


\begin{frame}{Aufgabe}
  \begin{itemize}
    \item<2-> a) Erl{"a}utere, wie man zu einem gegebenen Zeitpunkt t die Winkelweite $\alpha$ erh{"a}lt und umgekehrt. Bestimme f{"u}r t = 5 (t in h) den zugeh{"o}rigen Winkel.
    \item<3-> b) Mit welcher Funktion kann man zu einem gegebenen Winkel $\alpha$ den Wasserstand berechnen? Berechne den Wasserstand 5 Stunden nach Beobachtungsbeginn.
  \end{itemize}
\end{frame}


\begin{frame}{a) L{"o}sung}
  % \setlength{\abovedisplayskip}{0pt}
  % \setlength{\belowdisplayskip}{0pt}

  \onslide<1->{Erl{"a}utere, wie man zu einem gegebenen Zeitpunkt t die Winkelweite $\alpha$ erh{"a}lt und umgekehrt. Bestimme f{"u}r t = 5 (t in h) den zugeh{"o}rigen Winkel.}
  \begin{align}
    \onslide<2->{12h & \equalhat 360^\circ        & |:12 \tag{1} \\}
    \onslide<3->{1h  & \equalhat 30^\circ \tag{2} \\[-\baselineskip]\notag}
  \end{align}

  \onslide<4->{12h in \hyperref[fig:wasserstand]{Abbildung \autoref{fig:wasserstand}} entsprechen 360$^\circ$, also entspricht 1h dem Winkel 30$^\circ$.\\
  Daraus Folgt $\alpha = t \cdot 30^\circ$ (t in h).}
  \begin{align}
    \onslide<5->{\alpha & = t \cdot 30^\circ \nonumber \\}
    \onslide<6->{\alpha & = 5 \cdot 30^\circ \tag{1} \\}
    \onslide<7->{\alpha & = 150^\circ \tag{2} \\[-\baselineskip]\notag}
  \end{align}
  \onslide<8->{F{"u}r t = 5 erh{"a}lt man $\alpha = 150^\circ$}
\end{frame}

\begin{frame}{b) L{"o}sung}
  \only<1>{Mit welcher Funktion kann man zu einem gegebenen Winkel $\alpha$ den Wasserstand berechnen? Berechne den Wasserstand 5 Stunden nach Beobachtungsbeginn.}
  \only<1>{
    \begin{figure}[hb!]
      \centering
      \def\svgwidth{300px}
      \input{tmp/wasserstand.path.svg.pdf_tex}
      \caption{Wasserstand}
      \label{fig:wasserstand_two}
    \end{figure}
  }

  \onslide<2->{Da der Wasserstand zwischen -0,2 und 0,2 um den Durchschnittswert pendelt (\hyperref[fig:wasserstand_two]{Abbildung \autoref{fig:wasserstand_two}}), gilt:}

  \begin{align}
    \onslide<3->{f(\alpha) & =  sin(\alpha)} \nonumber \\
    \onslide<4->{f(\alpha) & = 1 \cdot sin(\alpha)}  \nonumber \\
    \onslide<5->{f(\alpha) & = \onslide<7->{0,2 \cdot} \onslide<6->{sin(\alpha)}  \nonumber \\[-\baselineskip]\notag}
  \end{align}

  \onslide<8->{F{"u}r t = 5:}
  \begin{align}
    \onslide<9->{\alpha         & = 5 \cdot 30^\circ          \tag{1} \\}
    \onslide<10->{\alpha        & = 150^\circ                 \tag{2} \\}
    \onslide<11->{f(\onslide<12->{150^\circ})  & = \onslide<13->{0,2 \cdot \sin(\onslide<14->{150^\circ})} \tag{3} \\}
    \onslide<15->{f(150^\circ)  & = 0,1                       \tag{4} \\[-\baselineskip]\notag}
  \end{align}
\end{frame}

\begin{frame}{b) Antwort}
  \onslide<1->{Nach 5 Stunden liegt der Wasserstand 10cm {"u}ber dem Durchschnittswert.}
  \onslide<1->{
    \begin{figure}[hb!]
      \centering
      \def\svgwidth{300px}
      \input{tmp/wasserstand_loesung.path.svg.pdf_tex}
      \caption{Wasserstand nach 5 Stunden}
      \label{fig:wasserstand_loesung}
    \end{figure}
  }
\end{frame} 

\end{document}