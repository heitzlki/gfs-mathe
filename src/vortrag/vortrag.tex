\documentclass[12pt,a4paper]{article}
\usepackage[utf8]{inputenc}
\usepackage[T1]{fontenc}
\usepackage[german]{babel} %deutsche Sprache
\usepackage{amsmath} %für Formeln
\usepackage{amsfonts} %Für Text in Formeln
\usepackage{amssymb} %Für mathematische Symbole
\usepackage{graphicx} %Für Bilder
\UseRawInputEncoding

\usepackage[absolute]{textpos}%Grafiken Positionieren

\usepackage{import} %für .pdf_tex Grafiken
\usepackage{color} %für .pdf_tex Grafiken
\usepackage{transparent} %für .pdf_tex Grafiken
\graphicspath{{tmp/}} %für .pdf_tex Grafiken
\usepackage{wrapfig} %Um Text neben Bildern zu positionieren

\usepackage{multicol} %Für mehr-spaltigen Text
\usepackage{xcolor,colortbl} %Für z.B Farben in Tabellen

\usepackage{hyperref} %Für Links wie z.B Abbildung X
\hypersetup{
	hidelinks
} %Um Link-Stile zu deaktivieren

\usepackage{adjustbox}%Um Elemente auf Seitenlänge zu bringen
\usepackage{standalone}%Um das Dokument aufzuteilen
\usepackage{ marvosym } %Füt lightning symbol
\usepackage{subcaption}
\captionsetup{compatibility=false}

\usepackage{scalerel,stackengine,amsmath} % Für ein "entspricht" Zeichen
\newcommand\equalhat{\mathrel{\stackon[1.5pt]{=}{\stretchto{%
    \scalerel*[\widthof{=}]{\wedge}{\rule{1ex}{3ex}}}{0.5ex}}}}

\usepackage[left=2cm,right=2cm,top=2cm,bottom=2cm]{geometry}  %Dokument Rand etc. einstellen
%\renewcommand{\familydefault}{\sfdefault} %sans-serif Schriftart
\title{Mit dem Sinus modellieren}
\author{Kirill Heitzler}
\date{\today}

\begin{document}

\maketitle
\begin{textblock}{210}(-1,4)
	\begin{figure}[h!]
		\input{tmp/title.path.svg.pdf_tex} %Grafik muss 220mm breit sein
	\end{figure}
\end{textblock}
\newpage

\tableofcontents
\newpage
\documentclass{standalone}
\begin{document}

\begin{frame}
  \begin{center}
    Grundlagen
  \end{center}
  \section{Grundlagen}
\end{frame}

\documentclass{standalone}

\begin{document}
\begin{frame}
  \begin{center}
    Rechtwinkliges Dreieck - Beschriftung
  \end{center}
  \subsection{Rechtwinkliges Dreieck - Beschriftung}
\end{frame}

\begin{frame}{Ecken}

  \only<1-5>{\begin{itemize}
    \item<2->Gegen den Uhrzeigersinn
    \item<3->A
    \item<4->B
    \item<5->C
  \end{itemize}}
  \noindent
  \only<6>{Die Ecken werden mit den Buchstaben A, B, C gegen den Uhrzeigersinn bei A angefangen beschriftet.}

  \only<1-2>{
    \begin{figure}[hb!]
      \centering
      \def\svgwidth{200px}
      \input{tmp/rechtwinkliges_dreieck.path.svg.pdf_tex}
      \caption{Rechtwinkliges Dreieck}
      \label{fig:rechtwinkliges_dreieck_ecken}
    \end{figure}
  }

  \only<3>{
    \begin{figure}[hb!]
      \centering
      \def\svgwidth{200px}
      \input{tmp/rechtwinkliges_dreieck.path.svg.pdf_tex}
      \caption{Rechtwinkliges Dreieck}
      \label{fig:rechtwinkliges_dreieck_ecke_A}
    \end{figure}
  }

  \only<4>{
    \begin{figure}[hb!]
      \centering
      \def\svgwidth{200px}
      \input{tmp/rechtwinkliges_dreieck.path.svg.pdf_tex}
      \caption{Rechtwinkliges Dreieck}
      \label{fig:rechtwinkliges_dreieck_ecke_B}
    \end{figure}
  }

  \only<5->{
    \begin{figure}[hb!]
      \centering
      \def\svgwidth{200px}
      \input{tmp/rechtwinkliges_dreieck.path.svg.pdf_tex}
      \caption{Rechtwinkliges Dreieck}
      \label{fig:rechtwinkliges_dreieck_ecke_C}
    \end{figure}
  }

\end{frame}

\begin{frame}{Winkel}
  \only<1-4>{\begin{itemize}
    \item<2->$\alpha$
    \item<3->$\beta$
    \item<4->$\gamma$
  \end{itemize}}
  \noindent
  \only<5>{Die Winkel $\alpha$, $\beta$, $\gamma$ werden in die Ecken der entsprechenden Buchstaben A, B, C gesetzt.}

  \only<1>{
    \begin{figure}[hb!]
      \centering
      \def\svgwidth{200px}
      \input{tmp/rechtwinkliges_dreieck.path.svg.pdf_tex}
      \caption{Rechtwinkliges Dreieck}
      \label{fig:rechtwinkliges_dreieck_winkel}
    \end{figure}
  }

  \only<2>{
    \begin{figure}[hb!]
      \centering
      \def\svgwidth{200px}
      \input{tmp/rechtwinkliges_dreieck.path.svg.pdf_tex}
      \caption{Rechtwinkliges Dreieck}
      \label{fig:rechtwinkliges_dreieck_ecken_winkel_alpha}
    \end{figure}
  }

  \only<3>{
    \begin{figure}[hb!]
      \centering
      \def\svgwidth{200px}
      \input{tmp/rechtwinkliges_dreieck.path.svg.pdf_tex}
      \caption{Rechtwinkliges Dreieck}
      \label{fig:rechtwinkliges_dreieck_ecken_winkel_beta}
    \end{figure}
  }

  \only<4->{
    \begin{figure}[hb!]
      \centering
      \def\svgwidth{200px}
      \input{tmp/rechtwinkliges_dreieck.path.svg.pdf_tex}
      \caption{Rechtwinkliges Dreieck}
      \label{fig:rechtwinkliges_dreieck_ecken_winkel_gamma}
    \end{figure}
  }
\end{frame}

\begin{frame}{Katheten}
    \only<1-3>{\begin{itemize}
      \item<2->"`Ankathete von $\alpha$"'
      \item<3->"`Gegenkathete von $\alpha$"'
    \end{itemize}}
    \noindent
    \only<4>{Die anliegende Kathete zu Winkel $\alpha$ wird "`Ankathete von $\alpha$"' genannt \pause und die Kathete gegen{"u}ber von $\alpha$ wird "`Gegenkathete von $\alpha$"' genannt.}
  
    \only<1>{
      \begin{figure}[hb!]
        \centering
        \def\svgwidth{200px}
        \input{tmp/rechtwinkliges_dreieck.path.svg.pdf_tex}
        \caption{Rechtwinkliges Dreieck}
        \label{fig:rechtwinkliges_dreieck_katheten}
      \end{figure}
    }
  
    \only<2>{
      \begin{figure}[hb!]
        \centering
        \def\svgwidth{200px}
        \input{tmp/rechtwinkliges_dreieck.path.svg.pdf_tex}
        \caption{Rechtwinkliges Dreieck}
        \label{fig:rechtwinkliges_dreieck_ankathete}
      \end{figure}
    }

    \only<3->{
      \begin{figure}[hb!]
        \centering
        \def\svgwidth{200px}
        \input{tmp/rechtwinkliges_dreieck.path.svg.pdf_tex}
        \caption{Rechtwinkliges Dreieck}
        \label{fig:rechtwinkliges_dreieck_gegenkathete}
      \end{figure}
    }
\end{frame}

\begin{frame}{Hypotenuse}
  \only<1-2>{\begin{itemize}
    \item<2->"`Hypotenuse"'
  \end{itemize}}
  \noindent
  \only<3>{Die Hypotenuse liegt gegen{"u}ber des rechten Winkels $\gamma$.}

  \only<1>{
    \begin{figure}[hb!]
      \centering
      \def\svgwidth{200px}
      \input{tmp/rechtwinkliges_dreieck.path.svg.pdf_tex}
      \caption{Rechtwinkliges Dreieck}
      \label{fig:rechtwinkliges_dreieck_hypotenuse_one}
    \end{figure}
  }

  \only<2->{
    \begin{figure}[hb!]
      \centering
      \def\svgwidth{200px}
      \input{tmp/rechtwinkliges_dreieck.path.svg.pdf_tex}
      \caption{Rechtwinkliges Dreieck}
      \label{fig:rechtwinkliges_dreieck_hypotenuse_two}
    \end{figure}
  }
\end{frame}

\end{document}


\documentclass{standalone}

\begin{document}
\begin{frame}
  \begin{center}
    Der Sinus
  \end{center}
  \subsection{Der Sinus}
\end{frame}

\begin{frame}{Definition}

  \onslide<1->{In einem rechtwinkligen Dreieck (\hyperref[fig:rechtwinkliges_dreieck_sinus_def]{Abbildung \autoref{fig:rechtwinkliges_dreieck_sinus_def}}) nennt man zu einem Winkel $\alpha$ des Dreiecks das Streckenverh{"a}ltnis}

  \begin{align}
    \onslide<2->{\sin(\alpha) & =} \onslide<3->{\dfrac{\onslide<3->{\text{Gegenkathete von $\alpha$}}}{\onslide<4->{\text{Hypotenuse}}}} \nonumber
  \end{align}

  \onslide<5->{\noindent den \textbf{Sinus von $\alpha$}.}

  \only<1-2>{
    \begin{figure}[hb!]
      \centering
      \def\svgwidth{200px}
      \input{tmp/rechtwinkliges_dreieck_sinus_def.path.svg.pdf_tex}
      \caption{Rechtwinkliges Dreieck}
      \label{fig:rechtwinkliges_dreieck_sinus_def}
    \end{figure}
  }

  \only<3>{
    \begin{figure}[hb!]
      \centering
      \def\svgwidth{200px}
      \input{tmp/rechtwinkliges_dreieck_sinus_def_gegenkathete.path.svg.pdf_tex}
      \caption{Rechtwinkliges Dreieck}
      \label{fig:rechtwinkliges_dreieck_sinus_def_gegenkathete}
    \end{figure}
  }

  \only<4>{
    \begin{figure}[hb!]
      \centering
      \def\svgwidth{200px}
      \input{tmp/rechtwinkliges_dreieck_sinus_def_hypotenuse.path.svg.pdf_tex}
      \caption{Rechtwinkliges Dreieck}
      \label{fig:rechtwinkliges_dreieck_sinus_def_hypotenuse}
    \end{figure}
  }

  \only<5->{
    \begin{figure}[hb!]
      \centering
      \def\svgwidth{200px}
      \input{tmp/rechtwinkliges_dreieck_sinus.path.svg.pdf_tex}
      \caption{Rechtwinkliges Dreieck}
      \label{fig:rechtwinkliges_dreieck_sinus}
    \end{figure}
  }
\end{frame}

\end{document}

\documentclass{standalone}

\begin{document}


\begin{frame}
  \begin{center}
    Sinus - Beispiel\\
    Gegenkathete von $\alpha$ mithilfe des Sinus berechnen
  \end{center}
  \subsection{Sinus - Beispiel}
\end{frame}

\begin{frame}{Aufgabe}
  Berechne die H{"o}he des Freiburger M{"u}nsters. Das rechtwinklige Dreieck in \hyperref[fig:rechtwinkliges_dreieck_am_muenster]{Abbildung \autoref{fig:rechtwinkliges_dreieck_am_muenster}} besitzt einen rechten Winkel (90$^\circ$), die Hypotenuse 164,05 Meter und die Winkelweite des Winkels $\alpha$ mit 45$^\circ$. Berechne die Gegenkathete von $\alpha$ namens x.\\
  \begin{figure}[hb!]
    \centeringf
    \centering
    \def\svgwidth{300px}
    \input{tmp/rechtwinkliges_dreieck_am_muenster.path.svg.pdf_tex}
    \caption{Rechtwinkliges Dreieck am M{"u}nster}
    \label{fig:rechtwinkliges_dreieck_am_muenster}
  \end{figure}
\end{frame}

\begin{frame}{Rechnung}
  \begin{itemize}
    \item<1-> $\alpha = 45^\circ$
    \item<2-> $\text{Hypotenuse} = 164,05m$
    \item<3-> Gegenkathete von $\alpha = x$
  \end{itemize}
  
  \begin{align}
    \onslide<4->{\sin(\alpha)                  & = \dfrac{\text{Gegenkathete von $\alpha$}}{\text{Hypotenuse}} \tag{1} \\}
    \onslide<5->{\sin(45^\circ )               & =} \onslide<6->{\dfrac{\onslide<6->{x}}{\onslide<7->{164,05m}}} & \onslide<8->{|\cdot 164,05m \tag{2} \\}
    \onslide<9->{\sin(45^\circ ) \cdot 164,05m & =} \onslide<10->{x \tag{3} \\}
    \onslide<11->{x                            & \cong} \onslide<12->{116m \tag{4} \\[-\baselineskip]\notag }
  \end{align}
\end{frame}

\begin{frame}{Antwort}
  \onslide<2->{Die Gegenkathete von $\alpha$ betr{"a}gt etwa 116 Meter, somit ist das M{"u}nster auch etwa 116 Meter gro{\ss}.}
  \onslide<1->{
    \begin{figure}[hb!]
      \centering
      \def\svgwidth{300px}
      \input{tmp/rechtwinkliges_dreieck_am_muenster_loesung.path.svg.pdf_tex}
      \caption{Rechtwinkliges Dreieck am M{"u}nster}
      \label{fig:rechtwinkliges_dreieck_am_muenster_loesung}
    \end{figure}
  }
\end{frame}

\end{document}

\documentclass{standalone}

\begin{document}

\begin{frame}
  \begin{center}
    Der Kosinus und der Tangens
  \end{center}
  \subsection{Der Kosinus und der Tangens}
\end{frame}

\begin{frame}{Sinus von $\alpha$}
  \begin{align}
    \onslide<1->{\sin(\alpha) & =} \onslide<2->{\dfrac{\onslide<2->{\text{Gegenkathete von $\alpha$}}}{\onslide<3->{\text{Hypotenuse}}}} \nonumber
  \end{align}

  \only<1>{
    \begin{figure}[hb!]
      \centering
      \def\svgwidth{200px}
      \input{tmp/rechtwinkliges_dreieck_sinus_def.path.svg.pdf_tex}
      \caption{Rechtwinkliges Dreieck}
      \label{fig:rechtwinkliges_dreieck_sinus_def_two}
    \end{figure}
  }

  \only<2>{
    \begin{figure}[hb!]
      \centering
      \def\svgwidth{200px}
      \input{tmp/rechtwinkliges_dreieck_sinus_def_gegenkathete.path.svg.pdf_tex}
      \caption{Rechtwinkliges Dreieck}
      \label{fig:rechtwinkliges_dreieck_sinus_def_gegenkathete}
    \end{figure}
  }

  \only<3->{
    \begin{figure}[hb!]
      \centering
      \def\svgwidth{200px}
      \input{tmp/rechtwinkliges_dreieck_sinus.path.svg.pdf_tex}
      \caption{Rechtwinkliges Dreieck}
      \label{fig:rechtwinkliges_dreieck_sinus}
    \end{figure}
  }
\end{frame}

\begin{frame}{Cosinus von $\alpha$}
  \begin{align}
    \onslide<1->{\cos(\alpha) & =} \onslide<2->{\dfrac{\onslide<2->{\text{Ankathete von $\alpha$}}}{\onslide<3->{\text{Hypotenuse}}}} \nonumber
  \end{align}
  
  \only<1>{
    \begin{figure}[hb!]
      \centering
      \def\svgwidth{200px}
      \input{tmp/rechtwinkliges_dreieck_sinus_def.path.svg.pdf_tex}
      \caption{Rechtwinkliges Dreieck}
      \label{fig:rechtwinkliges_dreieck_cosinus_def}
    \end{figure}
  }

  \only<2>{
    \begin{figure}[hb!]
      \centering
      \def\svgwidth{200px}
      \input{tmp/rechtwinkliges_dreieck_cosinus_ankathete.path.svg.pdf_tex}
      \caption{Rechtwinkliges Dreieck}
      \label{fig:rechtwinkliges_dreieck_cosinus_ankathete}
    \end{figure}
  }

  \only<3->{
    \begin{figure}[hb!]
      \centering
      \def\svgwidth{200px}
      \input{tmp/rechtwinkliges_dreieck_cosinus.path.svg.pdf_tex}
      \caption{Rechtwinkliges Dreieck}
      \label{fig:rechtwinkliges_dreieck_cosinus}
    \end{figure}
  }
\end{frame}

\begin{frame}{Tangens von $\alpha$}
  \begin{align}
    \onslide<1->{\tan(\alpha) & =} \onslide<2->{\dfrac{\onslide<2->{\text{Gegenkathete von $\alpha$}}}{\onslide<3->{\text{Ankathete von $\alpha$}}}} \nonumber
  \end{align}
  
  \only<1>{
    \begin{figure}[hb!]
      \centering
      \def\svgwidth{200px}
      \input{tmp/rechtwinkliges_dreieck_sinus_def.path.svg.pdf_tex}
      \caption{Rechtwinkliges Dreieck}
      \label{fig:rechtwinkliges_dreieck_tangens_def}
    \end{figure}
  }

  \only<2>{
    \begin{figure}[hb!]
      \centering
      \def\svgwidth{200px}
      \input{tmp/rechtwinkliges_dreieck_tangens_gegenkathete.path.svg.pdf_tex}
      \caption{Rechtwinkliges Dreieck}
      \label{fig:rechtwinkliges_dreieck_tangens_gegenkathete}
    \end{figure}
  }

  \only<3->{
    \begin{figure}[hb!]
      \centering
      \def\svgwidth{200px}
      \input{tmp/rechtwinkliges_dreieck_tangens.path.svg.pdf_tex}
      \caption{Rechtwinkliges Dreieck}
      \label{fig:rechtwinkliges_dreieck_tangens}
    \end{figure}
  }
\end{frame}

\end{document}


\end{document}

\documentclass{standalone}
\begin{document}

\begin{frame}
  \begin{center}
    Einheitskreis
  \end{center}
  \section{Einheitskreis}
\end{frame}

\documentclass{standalone}

\begin{document}

\subsection{Einheitskreis - Beispiel}

\textbf{Aufgaben-Text:} Auf einem kresif{"o}rmigen Koordinatensystem eines Radarschirms \autoref{fig:radar} wird die Lage von zwei Schiffen durch die Entfernung zum Hafen(0) und durch den Kurs gegen{"u}ber der x-Achse beschrieben. \\
\textbf{Aufgabe:} Ein Schiff \textbf{A} ist mit dem Kurs \textbf{30$^\circ$} gegen{"u}ber der x-Achse \textbf{einen Kilometer} weit gefahren. Welche Koordinaten im \textbf{x-y-Kooradinatensystem} hat es?\\
Welche Koordinaten hat das Schiff \textbf{B}, das mit dem Kurs \textbf{75$^\circ$} \textbf{einen Kilometer} weit gefahren ist?
\begin{figure}[hb!]
	\centering
	\def\svgwidth{250px}
	\input{tmp/Radar.path.svg.pdf_tex}
	\caption{Radar}
	\label{fig:radar}
\end{figure}
\\
\noindent
\textbf{L{"o}sung:}
\begin{wrapfigure}[9]{r}{0.5\textwidth}
	\def\svgwidth{250px}
	\input{tmp/radar_loesung.path.svg.pdf_tex}
	\caption{Radar L{"o}sung}
	\label{fig:radar_loesung}
\end{wrapfigure}
\\
Das Schiff \textbf{A} mit dem Kurs \textbf{30$^\circ$} befindet sich auf der x-Achse: etwa \textbf{0,86 Kilometer} und y-Achse: \textbf{0,5 Kilometer}. Also auf dem Punkt \textbf{A(0,86|0,5)} \\  \\
Das Schiff \textbf{B} mit dem Kurs \textbf{75$^\circ$} befindet sich auf der x-Achse: etwa \textbf{0,25 Kilometer} und y-Achse: \textbf{0,96 Kilometer}. Also auf dem Punkt \textbf{A(0,25|0,96)}

\end{document}

\input{modules/Einheitskreis/HypotenusenlängeEins.tex}
\documentclass{standalone}

\begin{document}

\begin{frame}
  \begin{center}
    Der Sinus und Kosinus am Einheitskreis
  \end{center}
  \subsection{Der Sinus und Kosinus am Einheitskreis}
\end{frame}

\begin{frame}{Dreieck mit Hypotenusenl{"a}nge 1}
	Dreiecke mit der \textbf{Hypotenusenl{"a}nge 1} kann man in einem Koordinatensystem auf folgenden Weise darstellen:
	\begin{figure}[hb!]
		\centering
		\def\svgwidth{200px}
		\input{tmp/rechtwinkliges_dreieck_sinus.path.svg.pdf_tex}
		\caption{Dreieck mit Hypotenusenl{"a}nge 1}
		\label{fig:rechtwinkliges_dreieck_sinus_def}
	\end{figure}
\end{frame}


\begin{frame}{1}
	Die Endpunkte der \textbf{Hypotenuse} sind der Ursprung O und ein Punkt \textbf{P}, der auf einem Kreis um O mit dem \textbf{Radius 1} liegt. Diesen Kreis nennt man den \textbf{Einheitskreis}.

	\begin{figure}[hb!]
		\centering
		\def\svgwidth{150px}
		\input{tmp/sin_cos_einheitskreis.path.svg.pdf_tex}
		\caption{Sinus und Kosinus am Einheitskreis}
		\label{fig:sin_cos_einheitskreis_1}
	\end{figure}
\end{frame}

\begin{frame}{2}
	Die Ecke mit dem rechten Winkel liegt auf der \textbf{x-Achse senkrecht unter P}. Der Punkt P hat somit Koordinaten \textbf{P(cos($\alpha$)|sin($\alpha$))}

	\begin{figure}[hb!]
		\centering
		\def\svgwidth{150px}
		\input{tmp/sin_cos_einheitskreis.path.svg.pdf_tex}
		\caption{Sinus und Kosinus am Einheitskreis}
		\label{fig:sin_cos_einheitskreis_2}
	\end{figure}
\end{frame}

\end{document}

\documentclass{standalone}

\begin{document}

\subsection{Beziehungen zwischen Sinus, Kosinus und Tangens}

\paragraph{1.}

\noindent F{"u}r $0^\circ < \alpha < 90^\circ$ nimmt $sin(\alpha)$ mit wachsendem $\alpha$ zu und $cos(\alpha)$ ab(\autoref{fig:0_alpha_90}). \\
$sin(0^\circ) = 0$, $cos(0^\circ) = 1$ (\autoref{fig:sin_null_cos_null_einheitskreis}), $sin(90^\circ) = 1$, $cos(90^\circ) = 0$ (\autoref{fig:sin_neunzig_cos_neunzig_einheitskreis}).

\newcommand{\smallImageSize}{130px} %Size for the Images saved in a variable
\begin{figure}[h!]
	\centering
	\begin{subfigure}[b]{0.3\linewidth}
		\def\svgwidth{\smallImageSize}
		\input{tmp/0_alpha_90.path.svg.pdf_tex}
		\caption{$0^\circ < \alpha < 90^\circ$}
		\label{fig:0_alpha_90}
	\end{subfigure}
	\begin{subfigure}[b]{0.3\linewidth}
		\def\svgwidth{\smallImageSize}
		\input{tmp/sin_null_cos_null_einheitskreis.path.svg.pdf_tex}
		\caption{$sin(0^\circ) = 0$, $cos(0^\circ) = 1$}
		\label{fig:sin_null_cos_null_einheitskreis}
	\end{subfigure}
	\begin{subfigure}[b]{0.3\linewidth}
		\def\svgwidth{\smallImageSize}
		\input{tmp/sin_neunzig_cos_neunzig_einheitskreis.path.svg.pdf_tex}
		\caption{$sin(90^\circ) = 1$, $cos(90^\circ) = 1$}
		\label{fig:sin_neunzig_cos_neunzig_einheitskreis}
	\end{subfigure}
	\caption{Beziehung 1}
	\label{fig:beziehung_eins}
\end{figure}

\newpage

\paragraph{2.}

Wendet man auf das im Einheitskreis dargestellte Dreieck den Satz des Pythagoras an(\autoref{fig:einheitskreis_dreieck_pythagoras}), so erh{"a}lt man den f{"u}r jede Winkelweite g{"u}ltigen Zusammenhang \\
$sin^2(\alpha) + cos^2(\alpha) = 1$. \\
Beispiel:

\begin{align}
	sin^2(\alpha) + cos^2(\alpha)                   & = 1 \tag{1}  \\
	(sin(45))^2 + (cos(45))^2                       & = 1  \tag{2} \\
	(\frac{\sqrt{2}}{2})^2 + (\frac{\sqrt{2}}{2})^2 & = 1 \tag{3}  \\
	\frac{\sqrt{2^2}}{2^2} + \frac{\sqrt{2^2}}{2^2} & = 1 \tag{4}  \\
	\frac{2}{4} + \frac{2}{4}                       & = 1 \tag{5}  \\
	\frac{1}{2} + \frac{1}{2}                       & = 1 \tag{6}  \\
	0,5 + 0,5                                       & = 1 \tag{7}
\end{align}

\begin{figure}[hb!]
	\centering
	\def\svgwidth{200px}
	\input{tmp/einheitskreis_dreieck_pythagoras.path.svg.pdf_tex}
	\caption{Einheitskreis Dreieck Satz des Pythagoras}
	\label{fig:einheitskreis_dreieck_pythagoras}
\end{figure}

\paragraph{3.}

\begin{wrapfigure}[6]{r}{0.45\textwidth}
	\def\svgwidth{200px}
	\input{tmp/sin_neunzig_minus_alpha_cos_neunzig_minus_alpha.path.svg.pdf_tex}
	\caption{sin(90$^\circ$ - $\alpha$); cos(90$^\circ$ - $\alpha$)}
	\label{fig:sin_neunzig_minus_alpha_cos_neunzig_minus_alpha}
\end{wrapfigure}

In \autoref{fig:sin_neunzig_minus_alpha_cos_neunzig_minus_alpha} sieht man: \\
$sin(90^\circ - \alpha) = x = cos(\alpha)$ und \\
$cos(90^\circ - \alpha) = y = sin(\alpha)$ \\
\textbf{Beispiel:}
\begin{align}
	sin(90^\circ - \alpha)   & = x                   & = cos(\alpha) \tag{1}   \\
	sin(90^\circ - 30^\circ) & =  \frac{\sqrt{3}}{2} & = cos(30^\circ) \tag{2}
\end{align}

\paragraph{4.}

Ebenfalls in \autoref{fig:sin_neunzig_minus_alpha_cos_neunzig_minus_alpha}:\\
$tan(\alpha) = \frac{y}{x} = \frac{sin(\alpha)}{\cos(\alpha)}$.

\end{document}

\documentclass{standalone}
\begin{document}

\subsection{Einheitskreis - Definition}
\textbf{Definition:} Es gilt:
\paragraph{}
$sin^2(\alpha) + cos^2(\alpha) = 1$
\paragraph{}
$sin(90^\circ - \alpha) = sin(\alpha)$
\paragraph{}
$tan(\alpha) = \frac{sin(\alpha)}{\cos(\alpha)}$, $\alpha \neq 90^\circ$, weil:
$tan(90)=\frac{sin(90)}{cos(90)} = \frac{1}{0} = $ \Lightning

\end{document}

\documentclass{standalone}

\begin{document}

\begin{frame}
  \begin{center}
    Einheitskreis - Aufgabe
  \end{center}
	\subsection{Einheitskreis - Aufgabe}
\end{frame}

\begin{frame}{Aufgabe}
	$\sin(\alpha) = 0,6$.\\
	\noindent\textbf{Bestimme:}
	\begin{itemize}
		\item<2-> a) $\cos(\alpha)$
		\item<3-> b) $\tan(\alpha)$
		\item<4-> c) $\sin(90^\circ - \alpha)$
		\item<5-> d) $\cos(90^\circ - \alpha)$
		\item<6-> e) $\tan(90^\circ - \alpha)$
	\end{itemize}
\end{frame}

\begin{frame}{a) L{"o}sung}
	\onslide<1->{
		$\sin(\alpha) = 0,6$\\
		$\cos(\alpha)$:
	}
	\begin{align}
		\onslide<2->{\sin^2(\alpha) + \cos^2(\alpha) & = 1 \tag{1} \\}
		\onslide<3->{0,6^2 +} \onslide<4->{cos^2(\alpha)          & = 1} \onslide<5-> {& |-0,6^2 \tag{2} \\}
		\onslide<6-> {\cos^2(\alpha)                 & = 1} \onslide<7->{ - 0,36} \onslide<8-> {& |\sqrt{} \tag{3} \\}
		\onslide<9->{\cos(\alpha)                   & = \sqrt{1 - 0,36}   \tag{4} \\}
		\onslide<10->{\cos(\alpha)                   & = \sqrt{0,64}   \tag{5} \\}
		\onslide<11->{\cos(\alpha)                   & = 0,8   \tag{6} \\ \nonumber}
	\end{align}
\end{frame}

\begin{frame}{b) L{"o}sung}
	\onslide<1->{
		$\sin(\alpha) = 0,6$\\
		$\cos(\alpha) = 0,8$\\
		$\tan(\alpha)$:
	}
	\begin{align}
		%		\onslide<7->{\tan(\alpha) & = \frac{3}{4}} \onslide<8->{ = 0,75 \tag{3} \\ \nonumber}
		\onslide<2->{\tan(\alpha) & = \frac{\sin(\alpha)}{\cos(\alpha)} \tag{1} \\}
		\onslide<3->{\tan(\alpha) & =} \onslide<4->{\frac{\onslide<4->{0,6}}{\onslide<5->{0,8}}} \onslide<6->{ = \frac{\frac{6}{10}}{\frac{8}{10}} \tag{2}  \\}
		\onslide<7->{\tan(\alpha) & =} \onslide<8->{\frac{\frac{6}{\cancel{10}}}{\frac{8}{\cancel{10}}} =} \onslide<9->{\frac{6}{8} \tag{3}  \\}
		\onslide<10->{\tan(\alpha) & = \frac{3}{4}} \onslide<11->{ = 0,75 \tag{4} \\ \nonumber}
	\end{align}
\end{frame}

\begin{frame}{c) L{"o}sung:}
	\onslide<1->{
		$\cos(\alpha) = 0,8$\\
		$\sin(90^\circ - \alpha)$:
	}
	\begin{align}
		\onslide<2->{sin(90^\circ  - \alpha) & =} \onslide<3->{cos(\alpha) =} \onslide<4->{0,8 \tag{1} \\ \nonumber}
	\end{align}
\end{frame}

\begin{frame}{d) L{"o}sung:}
	\onslide<1->{
		$\sin(\alpha) = 0,6$\\
		$\cos(90^\circ - \alpha)$:
	}
	\begin{align}
		\onslide<2->{cos(90^\circ  - \alpha) & =} \onslide<3->{sin(\alpha) =} \onslide<4->{0,6 \tag{1} \\ \nonumber}
	\end{align}
\end{frame}

\begin{frame}{e) L{"o}sung:}
	\onslide<1->{
		$\sin(90^\circ - \alpha) = 0,8$ \\
		$\cos(90^\circ - \alpha) = 0,6$ \\
		$\tan(90^\circ - \alpha)$:
	}
	\begin{align}
		\onslide<2->{tan(90^\circ  - \alpha) & =} \onslide<3->{\frac{\onslide<3->{sin(90^\circ  - \alpha)}}{\onslide<4->{cos(90^\circ  - \alpha)}}} \onslide<5->{ = \frac{0,8}{0,6} \tag{1} \\}
		\onslide<6->{tan(90^\circ  - \alpha) & =} \onslide<7->{\frac{8}{6} =} \onslide<8->{\frac{4}{3} \tag{2} \\ \nonumber}
	\end{align}
\end{frame}


\end{document}


\end{document}

\documentclass{standalone}
\begin{document}

\begin{frame}
  \begin{center}
    Mit dem Sinus modellieren
  \end{center}
  \section{Mit dem Sinus modellieren}
\end{frame}

\documentclass{standalone}
\begin{document}

\begin{frame}
  \begin{center}
    Mit dem Sinus modellieren - Beispiel
  \end{center}
  \subsection{Beispiel}
\end{frame}

\begin{frame}{Aufgabe}
  \onslide<1->{Bei einem Shaufelraddampfer dreht sich das Rad mit dem Durchmesser 2 Meter einmal vollst{"a}ndig in 60 Sekunden(\hyperref[fig:schaufelraddampfer]{Abbildung \autoref{fig:schaufelraddampfer}}). In welcher H{"o}he {"u}ber dem Wasserspiegel liegt der rot markierte Punkt A?\\
  Erstelle eine Wertetabelle in 5 Sekunden-Schritten.}

  \onslide<1->{
    \begin{figure}[hb!]
      \center
      \def\svgwidth{150px}
      \input{tmp/schaufelraddampfer.path.svg.pdf_tex}
      \caption{Schaufelraddampfer}
      \label{fig:schaufelraddampfer}
    \end{figure}
  }
  
  \definecolor{MyGrey}{rgb}{0.8,0.8,0.8}
  \onslide<2->{
    \begin{center}
      \begin{adjustbox}{width=0.8\textwidth}
        \begin{tabular}{ |>{\columncolor{MyGrey}}c|c|c|c|c|c|c|c|c|c|c|c|c|c| }
          \hline
          \rowcolor{MyGrey}
          Zeit t (in s)    & 0         & 5          & 10         & ... & 60          \\
          \hline
          Winkel $\alpha$  & 0$^\circ$ & 30$^\circ$ & 60$^\circ$ &     & 360$^\circ$ \\
          \hline
          H{"o}he h (in m) & 0         & 0,5        & 0,87       &     & 0           \\
          \hline
        \end{tabular}
      \end{adjustbox}
    \end{center}
  }
\end{frame}


\begin{frame}{L{"o}sung}
  \definecolor{MyGrey}{rgb}{0.8,0.8,0.8}
  \onslide<1->{
    \begin{center}
      \begin{adjustbox}{width=\textwidth}
        \begin{tabular}{ |>{\columncolor{MyGrey}}c|c|c|c|c|c|c|c| }
          \hline
          \rowcolor{MyGrey}
          Zeit t (in s)    & 0         & 5          & 10         & \onslide<2->{15}         & \onslide<3->{20}         & \onslide<4->{25}          & \onslide<1->{30}   \\
          \hline
          Winkel $\alpha$  & 0$^\circ$ & 30$^\circ$ & 60$^\circ$ & \onslide<5->{90$^\circ$} & \onslide<6->{120$^\circ$} & \onslide<7->{150$^\circ$} & \onslide<8->{180$^\circ$} \\
          \hline
          H{"o}he h (in m) & 0         & 0,5        & 0,87       & \onslide<9->{1}          & \onslide<10->{0,87}        & \onslide<11->{0,5}         & \onslide<12->{0}    \\
          \hline
        \end{tabular}
      \end{adjustbox}
    \end{center}
  }

  \only<1>{
    \begin{figure}[hb!]
      \center
      \def\svgwidth{150px}
      \input{tmp/schaufelraddampfer_60.path.svg.pdf_tex}
      \caption{Schaufelraddampfer}
      \label{fig:schaufelraddampfer_60}
    \end{figure}
  }

  \only<2-4>{
    \begin{figure}[hb!]
      \center
      \def\svgwidth{150px}
      \input{tmp/rad_60.path.svg.pdf_tex}
      \caption{Schaufelraddampfer}
      \label{fig:rad_60}
    \end{figure}
  }

  \only<5>{
    \begin{figure}[hb!]
      \center
      \def\svgwidth{150px}
      \input{tmp/rad_90.path.svg.pdf_tex}
      \caption{Schaufelraddampfer}
      \label{fig:rad_90}
    \end{figure}
  }

  \only<6>{
    \begin{figure}[hb!]
      \center
      \def\svgwidth{150px}
      \input{tmp/rad_120.path.svg.pdf_tex}
      \caption{Schaufelraddampfer}
      \label{fig:rad_120}
    \end{figure}
  }

  \only<7>{
    \begin{figure}[hb!]
      \center
      \def\svgwidth{150px}
      \input{tmp/rad_150.path.svg.pdf_tex}
      \caption{Schaufelraddampfer}
      \label{fig:rad_150}
    \end{figure}
  }

  \only<8>{
    \begin{figure}[hb!]
      \center
      \def\svgwidth{150px}
      \input{tmp/rad_180.path.svg.pdf_tex}
      \caption{Schaufelraddampfer}
      \label{fig:rad_180}
    \end{figure}
  }
%%%%%%%%%%%%%%%%%%%%%%%%%%%%%%%%%%%%%%%%%%%%%%%%%%%%%%%%%%%%%%%%
  \only<9>{
    \begin{figure}[hb!]
      \center
      \def\svgwidth{150px}
      \input{tmp/rad_90_sol.path.svg.pdf_tex}
      \caption{Schaufelraddampfer}
      \label{fig:rad_90_sol}
    \end{figure}
  }

  \only<10>{
    \begin{figure}[hb!]
      \center
      \def\svgwidth{150px}
      \input{tmp/rad_120_sol.path.svg.pdf_tex}
      \caption{Schaufelraddampfer}
      \label{fig:rad_120_sol}
    \end{figure}
  }

  \only<11>{
    \begin{figure}[hb!]
      \center
      \def\svgwidth{150px}
      \input{tmp/rad_150_sol.path.svg.pdf_tex}
      \caption{Schaufelraddampfer}
      \label{fig:rad_150_sol}
    \end{figure}
  }

  \only<12>{
    \begin{figure}[hb!]
      \center
      \def\svgwidth{150px}
      \input{tmp/rad_180_sol.path.svg.pdf_tex}
      \caption{Schaufelraddampfer}
      \label{fig:rad_180_sol}
    \end{figure}
  }

\end{frame}

\begin{frame}{L{"o}sung}
  \vspace{-0.5cm}
  \definecolor{MyGrey}{rgb}{0.8,0.8,0.8}

  \onslide<1->{\begin{center}
    \begin{adjustbox}{width=\textwidth}
      \begin{tabular}{ |>{\columncolor{MyGrey}}c|c|c|c|c|c|c|c| }
        \hline
        \rowcolor{MyGrey}
        Zeit t (in s) & 0 & 5 & 10 & 15 & 20 & 25 & 30 \\
        \hline
        Winkel $\alpha$  & 0$^\circ$ & 30$^\circ$ & 60$^\circ$ & 90$^\circ$ & 120$^\circ$ & 150$^\circ$ & 180$^\circ$ \\
        \hline
        H{"o}he h (in m) & 0 & 0,5 & 0,87 & 1 & 0,87 & 0,5 & 0 \\
        \hline
      \end{tabular}
    \end{adjustbox}
  \end{center}}

  \onslide<1->{
    \begin{center}
      \begin{adjustbox}{width=\textwidth}
        \begin{tabular}{ |>{\columncolor{MyGrey}}c|c|c|c|c|c|c|c| }
          \hline
          \rowcolor{MyGrey}
          Zeit t (in s)   & \onslide<1->{35}          & \onslide<2->{40}          & \onslide<3->{45}          & \onslide<4->{50}          & \onslide<5->{55}          & \onslide<6->{60}           \\
          \hline
          Winkel $\alpha$  & \onslide<7->{210$^\circ$} & \onslide<8->{240$^\circ$} & \onslide<9->{270$^\circ$} & \onslide<10->{300$^\circ$} & \onslide<11->{330$^\circ$} & \onslide<12->{360$^\circ$} \\
          \hline
          H{"o}he h (in m) & \onslide<13->{-0,5}        & \onslide<14->{-0,87}       & \onslide<15->{-1}          & \onslide<16->{-0,87}       & \onslide<17->{-0,5}        & \onslide<18->{0}           \\
          \hline
        \end{tabular}
      \end{adjustbox}
    \end{center}
  }
  \vspace{-0.25cm}

  \only<1-6>{
    \begin{figure}[hb!]
      \center
      \def\svgwidth{150px}
      \input{tmp/schaufelraddampfer_180.path.svg.pdf_tex}
      \caption{Schaufelraddampfer}
      \label{fig:schaufelraddampfer_180}
    \end{figure}
  }

  \only<7>{
    \begin{figure}[hb!]
      \center
      \def\svgwidth{150px}
      \input{tmp/rad_210.path.svg.pdf_tex}
      \caption{Schaufelraddampfer}
      \label{fig:rad_210}
    \end{figure}
  }

  \only<8>{
    \begin{figure}[hb!]
      \center
      \def\svgwidth{150px}
      \input{tmp/rad_240.path.svg.pdf_tex}
      \caption{Schaufelraddampfer}
      \label{fig:rad_240}
    \end{figure}
  }

  \only<9>{
    \begin{figure}[hb!]
      \center
      \def\svgwidth{150px}
      \input{tmp/rad_270.path.svg.pdf_tex}
      \caption{Schaufelraddampfer}
      \label{fig:rad_270}
    \end{figure}
  }

  \only<10>{
    \begin{figure}[hb!]
      \center
      \def\svgwidth{150px}
      \input{tmp/rad_300.path.svg.pdf_tex}
      \caption{Schaufelraddampfer}
      \label{fig:rad_300}
    \end{figure}
  }

  \only<11>{
    \begin{figure}[hb!]
      \center
      \def\svgwidth{150px}
      \input{tmp/rad_330.path.svg.pdf_tex}
      \caption{Schaufelraddampfer}
      \label{fig:rad_330}
    \end{figure}
  }

  \only<12>{
    \begin{figure}[hb!]
      \center
      \def\svgwidth{150px}
      \input{tmp/rad_360.path.svg.pdf_tex}
      \caption{Schaufelraddampfer}
      \label{fig:rad_360}
    \end{figure}
  }

  \only<13>{
    \begin{figure}[hb!]
      \center
      \def\svgwidth{150px}
      \input{tmp/rad_210_sol.path.svg.pdf_tex}
      \caption{Schaufelraddampfer}
      \label{fig:rad_210_sol}
    \end{figure}
  }

  \only<14>{
    \begin{figure}[hb!]
      \center
      \def\svgwidth{150px}
      \input{tmp/rad_240_sol.path.svg.pdf_tex}
      \caption{Schaufelraddampfer}
      \label{fig:rad_240_sol}
    \end{figure}
  }

  \only<15>{
    \begin{figure}[hb!]
      \center
      \def\svgwidth{150px}
      \input{tmp/rad_270_sol.path.svg.pdf_tex}
      \caption{Schaufelraddampfer}
      \label{fig:rad_270_sol}
    \end{figure}
  }

  \only<16>{
    \begin{figure}[hb!]
      \center
      \def\svgwidth{150px}
      \input{tmp/rad_300_sol.path.svg.pdf_tex}
      \caption{Schaufelraddampfer}
      \label{fig:rad_300_sol}
    \end{figure}
  }

  \only<17>{
    \begin{figure}[hb!]
      \center
      \def\svgwidth{150px}
      \input{tmp/rad_330_sol.path.svg.pdf_tex}
      \caption{Schaufelraddampfer}
      \label{fig:rad_330_sol}
    \end{figure}
  }

  \only<18>{
    \begin{figure}[hb!]
      \center
      \def\svgwidth{150px}
      \input{tmp/rad_360_sol.path.svg.pdf_tex}
      \caption{Schaufelraddampfer}
      \label{fig:rad_360_sol}
    \end{figure}
  }


\end{frame}


\begin{frame}{L{"o}sung}
  \definecolor{MyGrey}{rgb}{0.8,0.8,0.8}
  \onslide<1->{
    \begin{center}
      \begin{adjustbox}{width=\textwidth}
        \begin{tabular}{ |>{\columncolor{MyGrey}}c|c|c|c|c|c|c|c|c|c|c|c|c|c| }
          \hline
          \rowcolor{MyGrey}
          Zeit t (in s)    & 0         & 5          & 10         & 15         & 20          & 25          & 30          & 35          & 40          & 45          & 50          & 55          & 60          \\
          \hline
          Winkel $\alpha$  & 0$^\circ$ & 30$^\circ$ & 60$^\circ$ & 90$^\circ$ & 120$^\circ$ & 150$^\circ$ & 180$^\circ$ & 210$^\circ$ & 240$^\circ$ & 270$^\circ$ & 300$^\circ$ & 330$^\circ$ & 360$^\circ$ \\
          \hline
          H{"o}he h (in m) & 0         & 0,5        & 0,87       & 1          & 0,87        & 0,5         & 0           & -0,5        & -0,87       & -1          & -0,87       & -0,5        & 0           \\
          \hline
        \end{tabular}
      \end{adjustbox}
    \end{center}
  }

  \onslide<1->{
    \begin{figure}[hb!]
      \center
      \def\svgwidth{150px}
      \input{tmp/schaufelraddampfer_360.path.svg.pdf_tex}
      \caption{Schaufelraddampfer}
      \label{fig:schaufelraddampfer_360}
    \end{figure}
  }
\end{frame}


% \noindent\textbf{L{"o}sung:}\\



\end{document}
\documentclass{standalone}
\begin{document}

\begin{frame}
  \begin{center}
    Winkel $\alpha$ mit 0$^\circ$  $\leq$ $\alpha$ $\leq$ 90$^\circ$
  \end{center}
  \subsection{Winkel $\alpha$ mit 0$^\circ$  $\leq$ $\alpha$ $\leq$ 90$^\circ$}
\end{frame}

\begin{frame}{Winkel $\alpha$ mit 0$^\circ$  $\leq$ $\alpha$ $\leq$ 90$^\circ$}
  Am Einheitskreis entspricht sin($\alpha$) der y-Koordinate des Punktes P(\hyperref[fig:0_alpha_90_360]{Abbildung \autoref{fig:0_alpha_90_360}}).\\
  \textbf{$\sin(40^\circ ) \approx 0,64$}

  \begin{figure}[hb!]
    \center
    \def\svgwidth{150px}
    \input{tmp/0_alpha_90_360.path.svg.pdf_tex}
    \caption{Winkel $\alpha$ mit 0$^\circ$  $\leq$ $\alpha$ $\leq$ 90$^\circ$ }
    \label{fig:0_alpha_90_360}
  \end{figure}
\end{frame}

\end{document}
\documentclass{standalone}
\begin{document}

\subsection{Erweiterter Winkel $\alpha$ mit 90$^\circ$  $<$ $\alpha$ $\leq$ 360$^\circ$ }

\noindent Wird $\alpha$ {"u}ber 90$^\circ$  vergr{"o}{\ss}ert, wird der Sinuswert von $\alpha$ ebenso als y-Koordinate des Punktes P festgelegt(\autoref{fig:erweiterter_winkel_alpha}).

\begin{figure}[h!]
  \centering
  \begin{subfigure}[b]{0.5\linewidth}
    \def\svgwidth{220px}
    \input{tmp/sin_120_087.path.svg.pdf_tex}
    \caption{$\sin(120^\circ ) \approx 0,87$}
    \label{fig:sin_120_087}
  \end{subfigure}
  \begin{subfigure}[b]{0.35\linewidth}
    \def\svgwidth{220px}
    \input{tmp/sin_310_-_077.path.svg.pdf_tex}
    \caption{$\sin(310^\circ ) \approx -0,77$}
    \label{fig:sin_310_-_077}
  \end{subfigure}
  \caption{Erweiterter Winkel $\alpha$ mit 90$^\circ$  $<$ $\alpha$ $\leq$ 360$^\circ$ }
  \label{fig:erweiterter_winkel_alpha}
\end{figure}

\end{document}
\documentclass{standalone}
\begin{document}

\begin{frame}
  \begin{center}
    Erweiterter Winkel $\alpha$ mit 90$^\circ$  $<$ $\alpha$ $\leq$ 360$^\circ$  - Aufgabe
  \end{center}
  \subsection{Erweiterter Winkel $\alpha$ mit 90$^\circ$  $<$ $\alpha$ $\leq$ 360$^\circ$  - Aufgabe}
\end{frame}

\begin{frame}{Aufgabe}
  \onslide<1->{Ein Punkt P bewegt sich auf dem Einheitskreis(\hyperref[fig:alpha_0]{Abbildung \autoref{fig:alpha_0}}) gegen den Uhrzeigersinn. F{"u}r $\alpha$ = 0$^\circ$  befindet er sich im Punkt(1|0).}

  \onslide<2->{Bestimme}
  \onslide<3->{
    \begin{itemize}
      \item<3-> a) Gib die x- und y-Koordinaten des Punktes P f{"u}r $\alpha$ = 140$^\circ$  und f{"u}r $\alpha$ = 310$^\circ$ an.
      \item<4-> b) Bestimme zwei verschiedene Werte f{"u}r $\alpha$, sodass seine y-Koordinate 0,8 betr{"a}gt.
    \end{itemize}
  }
  
  \onslide<1->{
    \begin{figure}[hb!]
      \centering
      \def\svgwidth{100px}
      \input{tmp/alpha_0.path.svg.pdf_tex}
      \caption{$\alpha = 0^\circ $}
      \label{fig:alpha_0}
    \end{figure} 
  }
\end{frame}

\begin{frame}{a) L{"o}sung}\\
  \onslide<1->{F{"u}r $\alpha$ = 140$^\circ$}\onslide<4->{: Punkt (\onslide<7->{-0,77}|\onslide<4->{0,64})}
  \begin{align}
    \onslide<2->{\sin(\alpha)     & = y \tag{1}           \\}
    \onslide<3->{\sin(140^\circ ) & \approx} \onslide<4->{0,64 \tag{2}  \\}
    \onslide<5->{\cos(\alpha)     & = x \tag{3}           \\}
    \onslide<6->{\cos(140^\circ ) & \approx} \onslide<7->{-0,77 \tag{4} \\[-\baselineskip]\notag}
  \end{align}

  \onslide<8->{F{"u}r $\alpha$ = 310$^\circ$}\onslide<11->{: Punkt (\onslide<14->{0,64}|\onslide<11->{-0,77})}
  \begin{align}
    \onslide<9->{\sin(\alpha)     & = y \tag{1}           \\}
    \onslide<10->{\sin(310^\circ ) & \approx} \onslide<11->{-0,77 \tag{2} \\}
    \onslide<12->{\cos(\alpha)     & = x \tag{3}           \\}
    \onslide<13->{\cos(310^\circ ) & \approx} \onslide<14->{0,64 \tag{4} \\[-\baselineskip]\notag}
  \end{align}

\end{frame}

% \onslide<1->{F{"u}r $\alpha$ = 310$^\circ$}: Punkt (\onslide<7->{0,64}|\onslide<4->{-0,77})
% \begin{align}
%   \onslide<2->{\sin(\alpha)     & = y \tag{1}           \\}
%   \onslide<3->{\sin(310^\circ ) & \approx} \onslide<4->{-0,77 \tag{2} \\}
%   \onslide<5->{\cos(\alpha)     & = x \tag{3}           \\}
%   \onslide<6->{\cos(310^\circ ) & \approx} \onslide<7->{0,64 \tag{4} \\[-\baselineskip]\notag}
% \end{align}

\begin{frame}{b) L{"o}sung}\\
  \onslide<1->{Bestimme zwei verschiedene Werte f{"u}r $\alpha$, sodass seine y-Koordinate 0,8 betr{"a}gt.}\\
  \onslide<1->{F{"u}r $\alpha_1$}\onslide<4->{: $sin(53,1^\circ) \approx 0,8$}
  \begin{align}
    \onslide<2->{\sin^-^1(y) & = \alpha \tag{1}  \\}
    \onslide<3->{\sin^-^1(0,8) & \approx} \onslide<4->{53,1^\circ \tag{2}  \\[-\baselineskip]\notag}
  \end{align}

  \onslide<5->{F{"u}r $\alpha_2$}\onslide<8->{: $\sin(126,9^\circ) \approx 0,8$}
  \begin{align}
    \onslide<6->{\sin^-^1(0,8) & \approx 53,1^\circ \tag{1}  \\}
    \onslide<7->{53,1^\circ - 180^\circ & =} \onslide<8->{126,9^\circ \tag{2}  \\[-\baselineskip]\notag}
  \end{align}
\end{frame}

\end{document}
\documentclass{standalone}
\begin{document}

\subsection{Funktion f mit f($\alpha$)}

Ordnet man jedem Winkel $\alpha$ mit 0$^\circ$  $\leq \alpha \leq$ 360$^\circ$  seinen Sinuswert zu, so erh{"a}lt man eine Funktion f mit f($\alpha$) = sin($\alpha$).\\
Man kann mithilfe des Graphen von f (\autoref{fig:sinuswelle}) zu gegebenem Winkel den Sinuswert n{"a}herungsweise ablesen oder n{"a}herungsweise Winkel mit vorgegebenem Sinuswert ermitteln.

\begin{figure}[hb!]
  \center
  \def\svgwidth{500px}
  \input{tmp/sinuswelle.path.svg.pdf_tex}
  \caption{$f(\alpha) = \sin(\alpha)$}
  \label{fig:sinuswelle}
\end{figure}

\end{document}
\documentclass{standalone}
\begin{document}

\subsection{Sinusfunktion im Gradma{\ss} - Definition}

Die Funktion f mit f($\alpha$) = sin($\alpha$) hei{\ss}t \textbf{Sinusfunktion im Gradma{\ss}}.

\end{document}
\documentclass{standalone}
\begin{document}

\begin{frame}
  \begin{center}
    Graph einer Sinusfunktion zeichnen
  \end{center}
  \subsection{Graph einer Sinusfunktion zeichnen}
\end{frame}

\begin{frame}{Graph einer Sinusfunktion zeichnen}
  \definecolor{MyGrey}{rgb}{0.8,0.8,0.8}
  \onslide<1->{
    \begin{adjustbox}{width=\textwidth}
      \begin{tabular}{ |>{\columncolor{MyGrey}}c|c|c|c|c|c|c|c|c|c|c|c|c|c| }
        \hline
        Winkel $\alpha$ & 0$^\circ$ & 30$^\circ$ & 60$^\circ$ & 90$^\circ$ & 120$^\circ$ & 150$^\circ$ & 180$^\circ$ & 210$^\circ$ & 240$^\circ$ & 270^$\circ$ & 300$^\circ$ & 330$^\circ$ & 360$^\circ$ \\
        \hline
        $\sin(\alpha)$   & 0         & 0,5        & 0,87       & 1          & 0,87        & 0,5         & 0           & -0,5        & -0,87       & -1          & -0,87       & -0,5        & 0           \\
        \hline
      \end{tabular}
    \end{adjustbox}
  }

  \begin{figure}[hb!]
    \center
    \def\svgwidth{300px}

    \only<1>{\input{tmp/sinuswelle_zeichnen.path.svg.pdf_tex}}
    % \only<2>{\input{tmp/sinuswelle_zeichnen.path.svg.pdf_tex}}
    % \only<3>{\input{tmp/sinuswelle_zeichnen.path.svg.pdf_tex}}
    % \only<4>{\input{tmp/sinuswelle_zeichnen.path.svg.pdf_tex}}
    % \only<5>{\input{tmp/sinuswelle_zeichnen.path.svg.pdf_tex}}
    % \only<6>{\input{tmp/sinuswelle_zeichnen.path.svg.pdf_tex}}
    % \only<7>{\input{tmp/sinuswelle_zeichnen.path.svg.pdf_tex}}
    % \only<8>{\input{tmp/sinuswelle_zeichnen.path.svg.pdf_tex}}
    % \only<9>{\input{tmp/sinuswelle_zeichnen.path.svg.pdf_tex}}
    % \only<10>{\input{tmp/sinuswelle_zeichnen.path.svg.pdf_tex}}
    % \only<11>{\input{tmp/sinuswelle_zeichnen.path.svg.pdf_tex}}
    % \only<12>{\input{tmp/sinuswelle_zeichnen.path.svg.pdf_tex}}
    % \only<13>{\input{tmp/sinuswelle_zeichnen.path.svg.pdf_tex}}
    \caption{Sinuswelle Zeichnen}
    \label{fig:sinuswelle_zeichnen}
  \end{figure}

\end{frame}


\end{document}
\documentclass{standalone}
\begin{document}

\begin{frame}
  \begin{center}
    Einen Zeitlichen Vorgang modellieren
  \end{center}
  \subsection{Einen Zeitlichen Vorgang modellieren}
\end{frame}

\begin{frame}{Einen Zeitlichen Vorgang modellieren}
  \noindent In einem Hafenbecken schwankt der Wasserstand periodisch um seinen Durchschnittswert (\hyperref[fig:wasserstand]{Abbildung \autoref{fig:wasserstand}})

  \begin{figure}[hb!]
    \center
    \def\svgwidth{300px}
    \input{tmp/wasserstand.path.svg.pdf_tex}
    \caption{Wasserstand}
    \label{fig:wasserstand}
  \end{figure}
\end{frame}


\begin{frame}{Aufgabe}
  \begin{itemize}
    \item<2-> a) Erl{"a}utere, wie man zu einem gegebenenen Zeitpunkt t die Winkelweite $\alpha$ erh{"a}lt und umgekehrt. Bestimme f{"u}r t = 5 (t in h) den zugeh{"o}rigen Winkel.
    \item<3-> b) Mit welcher Funktion kann man zu einem gegebenen Winkel $\alpha$ den Wasserstand berechnen? Berechne den Wasserstand 5 Stunden nach Beobachtungsbeginn.
  \end{itemize}
\end{frame}


\begin{frame}{a) L{"o}sung}
  % \setlength{\abovedisplayskip}{0pt}
  % \setlength{\belowdisplayskip}{0pt}

  \onslide<1->{Erl{"a}utere, wie man zu einem gegebenenen Zeitpunkt t die Winkelweite $\alpha$ erh{"a}lt und umgekehrt. Bestimme f{"u}r t = 5 (t in h) den zugeh{"o}rigen Winkel.}
  \begin{align}
    \onslide<2->{12h & \equalhat 360^\circ        & |:12 \tag{1} \\}
    \onslide<3->{1h  & \equalhat 30^\circ \tag{2} \\[-\baselineskip]\notag}
  \end{align}

  \onslide<4->{12h in \hyperref[fig:wasserstand]{Abbildung \autoref{fig:wasserstand}} entsprechen 360$^\circ$, also entspricht 1h dem Winkel 30$^\circ$.\\
  Daraus Folgt $\alpha = t \cdot 30^\circ$ (t in h).}
  \begin{align}
    \onslide<5->{\alpha & = t \cdot 30^\circ \nonumber \\}
    \onslide<6->{\alpha & = 5 \cdot 30^\circ \tag{1} \\}
    \onslide<7->{\alpha & = 150^\circ \tag{2} \\[-\baselineskip]\notag}
  \end{align}
  \onslide<8->{F{"u}r t = 5 erh{"a}lt man $\alpha = 150^\circ$}
\end{frame}

\begin{frame}{b) L{"o}sung}
  \onslide<1->{Da der Wasserstand zwischen -0,2 und 0,2 um den Durchschnittswert pendelt (\autoref{fig:wasserstand}), gilt:}
  \begin{align}
    \onslide<2->{f(\alpha) & = 0,2 \cdot sin(\alpha)  \nonumber \\[-\baselineskip]\notag}
  \end{align}

  \onslide<3->{F{"u}r t = 5:}
  \begin{align}
    \onslide<4->{\alpha        & = 5 \cdot 30^\circ          \tag{1} \\}
                  \onslide<5->{& = 150^\circ                 \tag{2} \\}
    \onslide<6->{f(150^\circ)  & = 0,2 \cdot \sin(150^\circ) \tag{3} \\}
                  \onslide<7->{& = 0,1                       \tag{4} \\[-\baselineskip]\notag}
  \end{align}
\end{frame}

\begin{frame}{b) Antwort}
  \onslide<1->{Nach 5 Stunden liegt der Wasserstand 10cm {"u}ber dem Durchschnittswert.}
  \onslide<1->{
    \begin{figure}[hb!]
      \centering
      \def\svgwidth{300px}
      \input{tmp/wasserstand_loesung.path.svg.pdf_tex}
      \caption{Wasserstand nach 5 Stunden}
      \label{fig:wasserstand_loesung}
    \end{figure}
  }
\end{frame} 

\end{document}

\end{document}
\documentclass{standalone}
\begin{document}

\begin{frame}
  \begin{center}
    Anwendung
  \end{center}
  \section{Anwendung}
\end{frame}

\begin{frame}{Anwendung}
  \vspace{-\topsep}
  Auch wenn es uns nicht oft auff{"a}llt, viele technische Ger{"a}te bzw. Mechanismen verwenden die trigonometrischen Funktionen Sinus und Kosinus. Genauso wie viele mathematische Verfahren.\\
  \noindent Ein paar Beispiele:

  \begin{itemize}
    \setlength{\parskip}{0pt}
    \setlength{\itemsep}{0pt plus 1pt}
    \item<2-> Oszilloskop (elektronisches Messger{"a}t, das elektrische Spannungen in einen Verlaufsgraphen darstellt) 
    \item<3-> GPS - Global Positioning System (Positionsbestimmung mit Hilfe von Satelliten)
    \item<4-> Computergrafiken in 3D und 2D
    \item<5-> Landvermessungen
    \item<6-> Fourier Transformation (z. B. Anwendung beim Spektroskop f{"u}r Chemiker)
    \item<7-> Astronomen nutzten Spektroskope, um chemische Zusammensetzungen von weit entfernten Planeten zu bestimmen
  \end{itemize}
\end{frame}

\end{document}
\newpage
\documentclass{standalone}
\begin{document}

\section{Zusammenfassung}
\documentclass{standalone}
\begin{document}

\begin{frame}
  \begin{center}
    Einheitskreis - Zusammenfassung
  \end{center}
  \subsection{Einheitskreis - Zusammenfassung}
\end{frame}

\begin{frame}{Einheitskreis - Zusammenfassung}
  Die Endpunkte eines Dreickecks mit der Hypotenusenl{"a}nge 1 bilden den Ursprung 0 und einen Punkt P, der auf einem Kreis um 0 mit dem Radius 1 liegt und den Einheitskreis bildet.
  \begin{figure}[hb!]
    \center
    \def\svgwidth{150px}
    \input{tmp/0_alpha_90_360.path.svg.pdf_tex}
    \caption{Einheitskreis}
    \label{fig:0_alpha_90_360_two}
  \end{figure}
\end{frame}

\begin{frame}{Einheitskreis - Zusammenfassung}
  Die Gegenkathete l{"a}sst sich mit $\sin(\alpha)$ und die Ankathete mit $\cos(\alpha)$ berechnen.
  \begin{figure}[hb!]
    \centering
    \def\svgwidth{150px}
    \input{tmp/sin_cos_einheitskreis.path.svg.pdf_tex}
    \caption{Sinus und Kosinus am Einheitskreis}
    \label{fig:sin_cos_einheitskreis_two}
  \end{figure}
\end{frame}

\end{document}
\newpage
\documentclass{standalone}
\begin{document}

\subsection{Mit dem Sinus Modellieren - Zusammenfassung}
Ordnet man jedem Winkel $\alpha$ mit 0$^\circ$  $\leq \alpha \leq$ 360$^\circ$  seinen Sinuswert zu, so erh{"a}lt man eine Funktion f mit f($\alpha$) = sin($\alpha$).\\
Man kann mithilfe des Graphen von f (\autoref{fig:sinuswelle}) zu gegebenem Winkel den Sinuswert n{"a}herungsweise ablesen oder n{"a}herungsweise Winkel mit vorgegebenem Sinuswert ermitteln.

\begin{figure}[hb!]
  \center
  \def\svgwidth{500px}
  \input{tmp/sinuswelle.path.svg.pdf_tex}
  \caption{$f(\alpha) = \sin(\alpha)$}
  \label{fig:sinuswelle}
\end{figure}

Die Funktion f mit f($\alpha$) = sin($\alpha$) hei{\ss}t \textbf{Sinusfunktion im Gradma{\ss}}.

\end{document}

\end{document}
\newpage
\documentclass{standalone}
\begin{document}

\begin{frame}
  \begin{center}
    Quellen
  \end{center}
  \section{Quellen}
\end{frame}

\begin{frame}{Quellen}
  \begin{itemize}
    \item Freiburger M{"u}nster - \url{https://freiburg-schwarzwald.de/fotos06feb/freiburg12-060227.jpg}
    \item Vector Boot - \url{https://www.vecteezy.com/vector-art/170704-flat-ship-vectors}
    \item Lambacher Sweizer 9(S. 90 - 104) - Mathematik Buch
    \item Sinus und Kosinus im Alltag - \url{https://www.matheretter.de/wiki/sinus-kosinus-alltag}
  \end{itemize}
\end{frame}

\end{document}
\end{document}

